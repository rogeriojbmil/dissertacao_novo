%%%%%%%%%%%%%%%%%%%%%%%%%%%%%%%%%%%%%%%%
% Classe do documento
%%%%%%%%%%%%%%%%%%%%%%%%%%%%%%%%%%%%%%%%

% Nós usamos a classe "unb-cic".  Deixe apenas uma das linhas
% abaixo não-comentada, dependendo se você for do bacharelado ou
% da licenciatura.

% Para tirar os comentários, é só mudar o comando para fazer nada.
\newcommand{\com}[1]{\textcolor{red}{#1}}%

\documentclass[mpca]{unb-cic}



%%%%%%%%%%%%%%%%%%%%%%%%%%%%%%%%%%%%%%%%
% Pacotes importados
%%%%%%%%%%%%%%%%%%%%%%%%%%%%%%%%%%%%%%%%

\usepackage[brazil,american]{babel}
\usepackage[T1]{fontenc}
\usepackage{indentfirst}
\usepackage{natbib}
\usepackage{xcolor,graphicx,url}
\usepackage[utf8]{inputenc}
\usepackage{amsmath,amssymb,amsthm}
\usepackage{footnote}
%\usepackage{minipage}
\usepackage{tablefootnote}
\usepackage{listings}
\usepackage{enumerate}
\usepackage{graphicx} % pacote grafico
\usepackage{tabularx}
\usepackage{color}
\usepackage{multirow}
\usepackage{booktabs}
\usepackage{colortbl}
\usepackage{latexsym}
\usepackage{amssymb,proof}


%%%%%%%%%%%%%%%%%%%%%%%%%%%%%%%%%%%%%%%%
% Cores dos links
%%%%%%%%%%%%%%%%%%%%%%%%%%%%%%%%%%%%%%%%

% Veja o arquivos cores.tex se quiser ver que outras cores estão
% pré-definidas.  Utilizando o comando \hypersetup abaixo nós
% evitamos aquelas caixas vermelhas feias em volta dos links.

\input{cores}
\hypersetup{
  colorlinks=true,
  linkcolor=DarkScarletRed,
  citecolor=DarkScarletRed,
  filecolor=DarkScarletRed,
  urlcolor= DarkScarletRed
}



%%%%%%%%%%%%%%%%%%%%%%%%%%%%%%%%%%%%%%%%
% Informações sobre a monografia
%%%%%%%%%%%%%%%%%%%%%%%%%%%%%%%%%%%%%%%%
\title{UnB-CIC: Uma classe em LaTeX para textos do Departamento de Ciência da Computação}%

\orientador{\prof \dr Rodrigo Bonifácio de Almeida}{CIC/UnB}
\coorientador[a]{\prof[a] \dr[a] Edna Dias Canedo}{CIC/UnB}
\coordenador[a]{\prof \dr Marcelo Ladeira}{CIC/UnB}
\diamesano{24}{dezembro}{2014}%

\membrobanca{\prof[a] \dr[a] Membra da Banca}{MEC}
\membrobanca{\prof \dr Membro do Banco}{CIC/UnB}

\autor{Rogério Alves da.}{Conceição}
\CDU{004.4}

\palavraschave{Interoperabilidade, Segurança em arquiteturas orientadas a serviço, Web Services, Arquitetura de referência em SOA} % palavras-chave do trabalho
\keywords{Interoperability, Security in service-oriented architectures, Web Services, SOA Reference Architecture} % palavras-chave do trabalho em ingles



\graphicspath{{.}{img/}}%
\newcommand{\unbcic}{\texttt{UnB-CIC}}%
%%%%%%%%%%%%%%%%%%%%%%%%%%%%%%%%%%%%%%%%
% Texto
%%%%%%%%%%%%%%%%%%%%%%%%%%%%%%%%%%%%%%%%

\begin{document}
  \maketitle

  \begin{dedicatoria}
Citando o poeta: ``Eu dedico essa música a primeira garota que tá sentada ali na fila. Brigado!''
  \end{dedicatoria}

  \begin{agradecimentos}
Agradeço ao Prof. José Ralha, cujos esforços na versão anterior foram bem ``adaptados'' a este trabalho.
  \end{agradecimentos}

\hyphenation{au-xi-li-ar}

\begin{resumo}
  A DITEC, Divisão de Tecnologia da Polícia Civil do Distrito Federal, tem como responsabilidade estratégica o desenvolvimento dos softwares da instituição, muitas vezes apresentando necessidades de integração e compartilhamento de informações sensíveis com órgãos conveniados. Dada a criticidade desses sistemas e informações compartilhadas, preocupações relacionadas a segurança devem ser tratadas sob uma perspectiva arquitetural dentro da instituição, que atualmente adota diferentes alternativas de integração, desde \emph{Web Services} até a replicação das bases de dados para instituições parceiras. O objetivo desse trabalho é propor uma arquitetura orientada a serviços para ser adotada como alternativa única de integração, balanceando os requisitos de segurança com outros atributos de qualidade; em particular o tempo de processamento das requisições.
\end{resumo}

\selectlanguage{american}
\begin{abstract}
 DITEC, Technology Division Civil Police of the Federal District, is responsible for the software's institution strategic development, often presenting needs of integration and sharing of sensitive information with government insured. Given the criticality of these systems and shared information, security concerns should be treated under an architectural perspective within the institution, which currently adopts different integration alternatives, from \ emph {Web Services} to the replication of databases for partner institutions. The aim of this paper is to propose a service-oriented architecture to be adopted as an alternative single integration, balancing security requirements with other quality attributes, in particular the processing time of the requests.

\end{abstract}
\selectlanguage{brazil}
\hyphenation{a-tri-bu-tos}
  \tableofcontents
  \listoffigures
  \listoftables
  
  


\renewcommand{\appendixname}{Anexo}


  \textual

  %---------- Primeiro Capitulo ----------
\chapter{Introdução}\label{sec:introducao}
%\section{Apresentação}
As atribuições da Polícia Civil do Distrito Federal, no que diz respeito à sua competência de Polícia Judiciária, tangenciam em vários pontos as atribuições do Ministério Público do Distrito Federal e Territórios, do Tribunal de Justiça do Distrito Federal e Territórios e da Defensoria Pública do Distrito Federal. De forma que a competência de cada um desses Órgãos, por apresentarem pontos que se complementam, demanda intensa troca de informações.

A Polícia Civil do Distrito Federal, por meio de sua Divisão de Tecnologia, tem como propósito desenvolver seus próprios softwares. Esta atividade permite uma vantagem estratégica para a instituição, uma vez que a torna detentora dos softwares desenvolvidos evitando dessa forma a dependência tecnológica e administrativa de empresas privadas.

Nesse sentido, tem-se buscado estudar técnicas de desenvolvimento de software que promovam de forma efetiva a integração dos sistemas internos com os sistemas de Órgãos parceiros e que necessitem consumir de forma segura os dados e informações oriundos dos sistemas legados da Polícia Civil do Distrito Federal.

\section{Problema da pesquisa}

Nesse contexto, um dos principais desafios encontrados na DITEC refere-se à necessidade de integração e compartilhamento de informações de maneira segura, observando que as aplicações foram desenvolvidas em diferentes linguagens de programação e a integração ocorre com diferentes órgãos conveniados, tais como: Tribunal de Justiça do Distrito Federal e Território (TJDFT), Ministério Público da União (MPU), Departamento de Trânsito do DF (DETRAN-DF), Secretária de Segurança Pública do Distrito Federal (SSP-DF), Secretárias de Justiça do DF e Estados.

A ocorrência de uma vulnerabilidade de confidencialidade, por exemplo, ocorrendo o vazamento de informações sensíveis, criminosos poderiam utilizar essas informações e comprometer de forma significativa uma investigação policial.

Outra preocupação está relacionada à autenticidade, uma vez que todos os acessos a informações no âmbito da Polícia Civil do Distrito Federal devem ser realizados somente por pessoal autorizado. Caso isso não seja observado, pessoas podem se valer do anonimato e divulgar dados sigilosos de forma criminosa, o que também acarretaria inúmeros problemas de ordem jurídica para a instituição.

Dessa forma, devido à importância dessas informações, elas devem ter um tratamento diferenciado com relação a segurança nos aspectos de confidencialidade, autenticidade, integralidade e disponibilidade.

Por outro lado, na maioria das vezes, são disponibilizadas técnicas não seguras de integração, como a replicação ou o acesso direto a base de dados, apesar de existirem algumas iniciativas de integração baseadas em \emph{Web Services}.

No intuito de possibilitar que os sistemas possam ser integrados de forma eficiente e principalmente segura com outros sistemas, a Divisão de Tecnologia busca desenvolver uma metodologia própria que possa melhorar o processo integração de software no âmbito da Polícia Civil do Distrito Federal.
Para isso, optou-se pela utilização da Arquitetura Orientada a Serviços (SOA), que é um modelo arquitetural que propõem o uso de um conjunto de padrões para disponibilizar, descrever, publicar e invocar serviços. Neste cenário, este trabalho inicialmente propõe-se a investigar as seguintes questões de pesquisa:

\begin{enumerate}
	\item Quais são os principais problemas de segurança encontrados na adoção da Arquitetura Orientada a Serviços \-(SOA)? Essa questão de pesquisa é respondida nos capítulos 2 e 3 com o Mapeamento Sistemático e com a Revisão de Literatura.
	\item Quais padrões para construção de software seguro em arquiteturas SOA podem ser empregados pela Divisão de Tecnologia da Polícia Civil do Distrito Federal para realizar efetivamente a integração de seus sistemas com os sistemas dos órgãos parceiros? Neste caso, a resposta para essa questão é obtida no capítulo 3 com a Revisão de Literatura.
\end{enumerate}

Porém, posteriormente, após a proposição e o desenvolvimento do Protocolo de Autenticação e Autorização proposto, outras questões de pesquisa foram levantadas e investigadas. As questões são descritas a seguir:

\begin{enumerate}
\setcounter{enumi}{2}
  \item Qual o impacto observado no tempo de resposta às requisições com o uso do protocolo?
  \item Um protótipo funcional, sem foco em otimização, consegue suportar a demanda prevista?
\end{enumerate}  
  
Essas perguntas são respondidas no capítulo 5, com a realização de uma análise de desempenho do Protocolo de Autenticação e Autorização proposto.



\section{Justificativa}

Uma vez que a Polícia Civil do Distrito Federal desenvolva produtos de software mais seguros, que auxiliem no trabalho investigativo, ela realizará seu trabalho de uma forma mais efetiva, influenciando diretamente no combate da criminalidade e beneficiando a comunidade em geral e todos os órgãos distritais e federais tais como: Secretaria de Segurança Pública do Distrito Federal, Tribunal de Justiça do Distrito Federal, Ministério da Justiça, Secretarias de Governo Distritais, dentre outros órgãos, que necessitem das informações da instituição para realizar qualquer tipo de integração de software.

\section{Objetivos}\label{sec:Obj}
\subsection{Objetivo Geral}

Avaliar e aplicar o uso de técnicas, ferramentas  e procedimentos que garantam os requisitos de segurança em uma arquitetura orientada a serviços a ser usada para integrar os sistemas e automatizar os processos entre órgão parceiros (TJDFT, MPU, DETRAN, SSP).

\subsection{Objetivos Espec\'ificos}

\begin{enumerate}[a )]
	\item Realizar um mapeamento sistemático da literatura para compreender o estado da arte e da prática de segurança em SOA;

	\item Identificar e avaliar quais são os principais problemas de segurança encontrados na adoção da Arquitetura Orientada a Serviços (SOA);

	\item Estudar as especificações de Web Services relacionados a segurança e selecionar padrões e ferramentas para garantir confidencialidade, autenticidade e integridade nas integrações da arquitetura orientada a serviços. Essa seleção deve considerar o impacto na disponibilidade e no tempo de resposta dos serviços;

    \item Estabelecer uma arquitetura de referência na construção de software seguro em  SOA, por meio de um protocolo de autenticação e autorização, que possa ser empregado pela Divisão de Tecnologia da Polícia Civil do Distrito Federal para realizar efetivamente a integração de seus sistemas com os sistemas dos órgãos parceiros.

\end{enumerate}

\section{Organização do Trabalho}

Este trabalho está organizado em seis capítulos. No capítulo 2 é apresentado um mapeamento sistemático e os resultados obtidos com a sua realização. No capítulo 3 é realizada uma revisão da literatura onde são abordados os conceitos gerais sobre Arquitetura Orientada a Serviços (SOA), Web Services, REST, segurança e vulnerabilidades em SOA. Além disso, também são apresentados alguns protocolos de autenticação e autorização. No capítulo 4 é apresentado o protocolo de autenticação e autorização proposto e objeto deste trabalho. Neste capítulo são descritos os requisitos e a arquitetura do protocolo. É realizada uma análise formal do protocolo utilizando-se a lógica BAN. Neste capítulo também e descrita a implementação de um protótipo do protocolo proposto bem como uma análise de segurança. No capítulo 5 é realizada uma avaliação e análise de desempenho e são apresentados os resultados dos experimentos realizados. No capítulo 6 são apresentadas as conclusões do trabalho, bem como trabalhos futuros. 
  \input{tex/Capitulo2}
  \input{tex/Capitulo3}
  \chapter{Protocolo de Autenticação e Autorização proposto}\label{cap:Protocolo}
\section{Introdução}
%A Polícia Civil do Distrito Federal, diante da necessidade de compartilhar suas informações com órgão parceiros, no intuito de possibilitar que os sistemas possam ser integrados de forma eficiente e principalmente segura busca estabelecer uma arquitetura de referência para a adoção de uma arquitetura orientada a serviços. Essa arquitetura deve primar pela segurança, haja vista a criticidade e sensibilidade das informações que são tratadas no âmbito da PCDF.

%Dessa forma, optou-se por adotar a tecnologia de Web Services usando o protocolo REST para implementar SOA na instituição. Neste caso, estudos específicos foram realizados com vistas a estabelecer uma política de segurança eficiente que possibilite o fornecimento dos serviços e promova a integração com os órgãos parceiros.
A Polícia Civil do Distrito Federal, diante da necessidade de compartilhar suas informações com órgão parceiros, no intuito de possibilitar que os sistemas possam ser integrados de forma eficiente e principalmente segura busca estabelecer uma arquitetura de referência para a adoção de uma arquitetura orientada a serviços. Essa arquitetura deve primar pela segurança, haja vista a criticidade e sensibilidade das informações que são tratadas no âmbito da PCDF.

Dessa forma, optou-se por adotar a Web services \emph{RESTFull} para implementar SOA na Instituição. Neste caso, estudos específicos foram realizados com vistas a estabelecer uma política de segurança eficiente que possibilite o fornecimento dos serviços e promova a integração com os órgãos parceiros. Sendo assim, surgiu a necessidade da criação de um protocolo seguro e personalizado de autenticação e autorização que atenda às necessidades da PCDF.

\section{Requisitos do Protocolo}\label{sec:reqprotocolo}

O protocolo de autenticação e autorização proposto deverá ser aderente a arquitetura REST, de forma que possa permitir que os serviços ofertados pela Divisão de Tecnologia possam ser acessados por um número relativamente grande de clientes. Os requisitos inerentes ao protocolo são descritos nesta seção.

\begin{enumerate}[RQ1]

\item Para promover a segurança de sessão, toda comunicação entre o cliente e o servidor será realizada utilizando HTTPS \emph{(Hypertext Transfer Protocol over Secure Sockets Layer)}, usando o SSL/TLS para garantir a confidencialidade e integridade para a sessão. Para isso será usado o certificado digital X.509, emitido por uma autoridade de certificação, para encriptar as comunicações e garantir a autenticidade do servidor e do cliente. Devendo os clientes realizar a validação do certificado antes de interagir com o servidor.

\item Será utilizado a criptografia assimétrica para promover a segurança na troca de mensagens realizada entre o Cliente e a PCDF. Todas as mensagens deverão ser assinadas digitalmente. Para isso, será utilizado uma função hash, com o algoritmo SHA(Secure Hash Algorithm).

\item O protocolo deverá permitir acesso aos serviços apenas ao pessoal autorizado, de forma que a autenticação e autorização, siga padrões definidos na política de segurança. Sendo que para ser autenticado e autorizado o usuário deverá apresentar credenciais válidas. Essas credenciais deverão ser criptografadas, assinadas e enviadas no cabeçalho do protocolo HTTPS. Devendo ser escalável em termos de sobrecarga, tamanho do domínio de proteção e de manutenção. Além disso, ele deverá permitir a preservação de privacidade, uma vez que para proteger os clientes e fornecedores de recursos de entidades maliciosas, suas interações deverão revelar o mínimo de informações possíveis.

\item A autenticação e autorização será baseada em desafios e resposta, que serão elaborados a partir da apresentação de declarações de identidade (Claims). Tal requisito torna mais flexível o gerenciamento da identidade do usuário, uma vez que possibilita ao administrador desabilitar credenciais que tenham sido comprometidas de forma transparente ao usuário.

\item A política de autenticação e autorização proposta no protocolo será estabelecida por meio de contrato onde serão definidos todas as regras que deverão ser atendidas pelos usuário e pelo fornecedor do serviço.

    Dessa forma, para que qualquer usuário possa ter acesso aos serviços ofertados pela Divisão de Tecnologia da PCDF ele deverá concordar com um contrato prévio de acesso. Devendo primeiramente ser cadastrado e ter definido quais são seus privilégios de acesso/autorização. Uma vez cadastrado o usuário deverá informar os dados que possam comprovar sua identidade no momento da autenticação de forma que ele possa ser autorizado de acordo com o seus privilégios.

    No momento do credenciamento será gerado para o cliente múltiplas credenciais, que serão utilizadas no processo de autenticação e autorização, essas informações serão compartilhados entre o Cliente o Servidor de Autenticação e Autorização e o Servidor REST. Além disso, o Contrato poderá ter acesso a múltiplos serviços. A Figura ~\ref{fig:diagrama_relacionamento} apresenta o relacionamento entre o contrato, credenciais e serviços.

\begin{figure}[!htb]
    \centering
    \includegraphics[width=0.8\textwidth]{modelo_relacionamento_contrato1.png}
    \caption{Diagrama de relacionamento entre contrato, credenciais e usuários}
    \label{fig:diagrama_relacionamento}
\end{figure}

\end{enumerate}


%Para a implementação de segurança em aplicações REST, verificou-se que ela passa basicamente pela aplicação de segurança em protocolos \emph{HTTP}, que oferece dois tipos de autenticação:  \emph{Basic} e \emph{Digest}.

%A autenticação Basic é um modelo baseado no desafio e resposta, sendo utilizada por servidores HTTP para validar a autenticação~\cite{franks1999}. Desta forma, quando o cliente tenta acessar algum recurso protegido, a sua identidade é requerida pelo servidor, o cliente então fornece a resposta codificada em base64 no header \emph{HTTP},  se a resposta for correta ela terá acesso ao sistema. Porém, por não criptografar o desafio, estando esse apenas codificado, faz com que ele seja vulnerável e sujeito a ataques, como por exemplo, os de repetição.

%Já na Digest, o processo é o mesmo que na autenticação básica. Sendo que seu mecanismo de autenticação é um pouco mais complexo, uma vez que ele gera um HASH, geralmente utilizando o algoritmo MD5, do desafio que será enviado pelo servidor ao cliente ~\cite{franks1999}. Apesar de ser mais seguro do que a autenticação básica, autenticação HTTP Digest também é vulnerável à ataques, como por exemplo o man-in-the-middle.  Para evitar esse problema, deve ser empregado a segurança na camada de transporte~\cite{Webber10}.

\section{Arquitetura do Protoloco}\label{sec:ArqProtocolo}

A arquitetura do protocolo proposto é apresentada na Figura~\ref{fig:arquiteturaprotocolo}. O protocolo é composto por quatro componentes: Cliente, Servidor de Autenticação e Autorização, Servidor REST, e Banco de Dados de Autenticação e Autorização.

\begin{figure}[!htb]
    \centering
    \includegraphics[width=0.8\textwidth]{arquitetura_protocolo.png}
    \caption{Fluxo do protocolo de autenticação/autorização proposto, 1º cenário.}
    \label{fig:arquiteturaprotocolo}
\end{figure}

Os componentes da arquitetura são detalhados a seguir.

Cliente: O componente Cliente na arquitetura do protocolo representa as Instituições ou Órgãos conveniados, que após firmar um contrato, podem consumir os serviços ofertados pela PCDF.

Servidor de Autenticação e Autorização: Este componente tem um papel fundamental na arquitetura do protocolo, pois é nele que o gerenciamento de autenticação e autorização é realizado. Desta forma, o servidor de Autenticação e Autorização é responsável por realizar os processos de verificação e validação de credenciais, criação dos desafios de autenticação, criação de \emph{tokens} JSON e a criação e o gerenciamento das credenciais de autenticação e autorização temporárias, que são utilizados pelos clientes para consumir os serviços requisitados.

Servidor REST: Esse é um servidor de fachada que abstrai toda lógica necessária para o consumo dos serviços.Neste servidor estão concentrados os serviços REST disponibilizados pela PCDF. Desta forma, quando um Cliente necessita acessar um serviço, primeiramente ele deve ser autenticado e autorizado no servidor de Autenticação e Autorização. Após esse processo, o Cliente faz a requisição ao servidor de REST, que realiza as verificações necessárias para saber se o Cliente tem privilégios ou não para acessar o serviço. Ele acessa a base de dados de Autenticação e Autorização para confirmar as credenciais de autenticação e autorização temporária informadas e caso elas sejam válidas ele permite que o Cliente acesse o serviço requerido. Um ponto importante a ser destacado é que os desenvolvedores, ao desenvolver um serviço, não necessitam ter preocupações de segurança, uma vez que é esse componente que realiza esse atividade.

Banco de Autenticação e Autorização: Servidor de banco de dados que contém a estrutura de banco de dados necessário  para o funcionamento dos serviços de autenticação e autorização. É neste servidor que são gravados as credenciais, usuários, os desafios e as credenciais de autorização e autenticação temporária.


\subsection{Visão geral do protocolo de Autenticação e Autorização proposto}

Para ter acesso a API \emph{REST}, referente aos serviços ofertados, o cliente deverá ser autenticado e autorizado a acessar o serviço. Para isso, será usado a autenticação baseada em \emph{tokens} de segurança, que são recipientes de reivindicações da autoridade emissora. Os \emph{tokens} de segurança utilizados serão os \emph{Web Tokens} no formato \emph{JSON}. Esse, ao contrário dos tokens \emph{SAML}, que são baseados em \emph{XML}, são mais compactos e, portanto mais adequados para serem usados em um cabeçalho \emph{HTTP}. Além disso, todas as mensagens deverão ser assinadas e criptografadas de forma assimétrica. O processo de autenticação e autorização é descrito em dois cenários distintos. No primeiro cenário, representado na figura ~\ref{fig:protocoloseguro}, o Cliente, não está autenticado. No segundo cenário, ele está autenticado e possui uma credencial de autorização. %Esse último é representado na figura ~\ref{fig:cenario2}.

\subsubsection{Primeiro cenário}
No primeiro cenário, o cliente não está autenticado, e irá solicitar a autenticação pela primeira vez, conforme descrito a seguir:
\newline
\begin{figure}[!htb]
    \centering
    \includegraphics[width=1.0\textwidth]{fluxo_autenticacao.png}
    \caption{Fluxo do protocolo de autenticação/autorização proposto, 1º cenário.}
    \label{fig:protocoloseguro}
\end{figure}


O protocolo tem início quando o Cliente envia uma solicitação de autenticação ao servidor de Autenticação e Autorização. Esse pedido é realizado por meio de uma mensagem (mensagem 1 da Figura~\ref{fig:protocoloseguro}) que contém um \emph{token} JSON, enviado no cabeçalho HTTP da requisição REST. O \emph{token} contém uma credencial, extraída de forma aleatória da tabela de credenciais do Cliente. O token é assinado digitalmente pelo Cliente e cifrado com a chave pública do servidor de Autenticação e Autorização. É importante frisar que tanto o Cliente quanto o servidor de Autenticação e Autorização possuem as mesmas tabelas de credenciais e de serviços, pois elas são geradas no momento de assinatura do contrato de prestação do serviço.

Na segunda mensagem, ao receber uma solicitação de autenticação, o servidor de Autenticação e Autorização extrai o token cifrado com sua chave privada e verifica a autenticidade e integridade da requisição por meio da verificação da assinatura digital do Cliente.  Se houver qualquer problema, uma mensagem de erro HTTP (código 401): usuário não autorizado, é retornada ao Cliente.
Outra verificação que é realizada é a do \emph{timestamp}, que se refere ao tempo de envio da mensagem, se ela tiver sido enviada em um período de tempo superior ao pré-estabelecido no contrato, o Cliente também recebe uma mensagem de erro HTTP de usuário não autenticado.

Caso não haja problemas, procede-se com o processo de validação da credencial informada, que consiste em consultar a credencial em uma base de dados e se a credencial for válida e estiver associada ao Cliente, o Servidor de Autenticação e Autorização gera um desafio de autenticação. Tal desafio consiste em fazer uma busca aleatória à tabela de credenciais e selecionar um código de credencial que esteja associado ao Cliente. Em seguida grava-se o desafio, a data e hora de geração do desafio e a resposta que o Cliente deverá fornecer. um \emph{token} JSON, contendo o código do desafio, o código da credencial e um \emph{timestamp} representando a data e hora de criação do desafio, é enviado ao Cliente, assinado digitalmente pelo servidor de Autenticação e Autorização e cifrado com a chave pública do Cliente que está solicitando a autenticação.

Na terceira mensagem, após receber o desafio do servidor de Autenticação e Autorização, o Cliente, extrai o token cifrado com sua chave privada e verifica a autenticidade e integridade da requisição por meio da verificação da assinatura digital do servidor de Autenticação e Autorização. Em seguida verifica o \emph{timestamp}, cujo objetivo é o de verificar se a mensagem foi enviada em um período de tempo superior ao pré-estabelecido no contrato. Se houver qualquer problema, o processo de autenticação atual é descartado e inicia-se um novo processo de autenticação.

Caso não haja problemas, o Cliente verifica e responde o desafio solicitado, enviando-o, juntamente com um \emph{timestamp} e o código do serviço, que ele deseja consumir, para o Servidor de Autenticação e Autorização por meio de um \emph{token} JSON, que é assinado digitalmente pelo Cliente e cifrado com a chave pública do servidor de Autenticação e Autorização.

Na quarta mensagem, o servidor de Autenticação e Autorização recebe a resposta do desafio de autenticação, decifra o token e verifica a autenticidade e integridade da requisição por meio da verificação da assinatura digital do Cliente.  Não ocorrendo problemas, inicia-se o processo de verificação da resposta. A primeira verificação que é realizada refere-se ao tempo de geração do desafio, por meio do \emph{timestamp}. Se a resposta tiver sido enviada em um período de tempo superior ao pré-estabelecido em contrato, o servidor de Autenticação e Autorização envia uma mensagem de erro HTTP (código 401): usuário não autorizado, ao Cliente. Caso contrário, ele procede com a verificação do desafio que consiste em realizar uma consulta na tabela de desafios verificando se a resposta dada é a mesma que a esperada. Caso a resposta esteja correta o servidor de Autenticação e Autorização autentica o Cliente. Em seguida ele verifica, pelo código do serviço requisitado se o Cliente tem privilégios necessários para consumir o serviço requisitado.

Se a resposta for positiva, o servidor de Autenticação e Autorização gera uma credencial de autenticação e autorização temporária para o serviço solicitado. Ela é gravada em uma tabela de credencias de autorização temporária juntamente com a data e hora de criação, data de expiração e o código do cliente. A tabela de credencias de autorização temporária será acessada pelo Servidor REST  para verificar quais privilégios a entidade requisitante do serviço tem acesso e se ela está autenticada. O token, contendo a credencial de autenticação e autorização temporária, é  assinado digitalmente pelo servidor de Autenticação e Autorização   e cifrado  com a chave pública do Cliente. Após esse processo ele é enviado ao Cliente.

Caso a resposta do desafio esteja em desacordo com a esperada ou se o Cliente não tiver privilégios suficientes para acessar o serviço requisitado, ele recebe uma mensagem de erro HTTP (código 401): usuário não autorizado.
É importante frisar que a credencial de autenticação e autorização temporária será gerada apenas para o serviço que o Cliente tenha solicitado e possua o privilégio de acesso para utilizá-la. Ela será válida por um período  de tempo que será definido no momento da assinatura do contrato de prestação de serviço, entre o órgão conveniado e a PCDF.

Na quinta mensagem, o Cliente, extrai o token cifrado com sua chave privada e verifica a autenticidade e integridade da requisição por meio da verificação da assinatura digital do servidor de Autenticação e Autorização. Em seguida verifica o \emph{timestamp}, cujo objetivo é o de verificar se a mensagem foi enviada em um período de tempo superior ao pré-estabelecido no contrato. Se houver qualquer problema, o processo de autenticação atual é descartado e inicia-se um novo processo de autenticação.

Caso não haja problemas, o Cliente verifica a data e hora de validade da credencial de autorização temporária para saber se ela é válida. Confirmada sua validade, ele envia ao servidor REST, a requisição do serviço que deseja consumir juntamente com a credencial de autenticação e autorização temporária. O token de autenticação e autorização temporária é assinado com a chave privada do Cliente e cifrado com chave pública do servidor REST, sendo enviado no cabeçalho da requisição.

Finalmente, após receber a requisição, o Servidor REST, extrai o token cifrado com sua chave privada e verifica a autenticidade e integridade da requisição por meio da verificação da assinatura digital do servidor de Autenticação e Autorização. Em seguida verifica o \emph{timestamp}, para saber se a mensagem foi enviada em um período de tempo superior ao pré-estabelecido no contrato. Não havendo problemas, o Servidor REST verifica se a credencial de autenticação e autorização temporária é valida. Para isso, ele realiza uma consulta na tabela de credenciais temporárias, com a finalidade de confirmar se a credencial informada não expirou, se  foi realmente gerada para o Cliente e se ela está associada ao serviço solicitado.

Caso não haja problemas, o Cliente recebe os dados referentes à sua requisição. Havendo qualquer problema ele recebe uma mensagem de erro HTTP (código 401) de usuário não autorizado.

%Já no segundo cenário, que é representado na figura ~\ref{fig:cenario2}. O Cliente, já possui uma credencial de autorização temporária, neste caso ele deverá apresentá-la sempre que desejar consumir algum serviço que ele tenha acesso.
\subsubsection{Segundo cenário}

Já no segundo cenário, o Cliente, já possui uma credencial de autorização temporária, neste caso ele deverá apresentá-la sempre que desejar consumir algum serviço que ele tenha acesso conforme descrito a seguir:

Neste caso, o Cliente envia um token contendo uma credencial temporária no cabeçalho da requisição do serviço que deseja consumir, ao Servidor de REST.

O servidor de REST, recebe o token de autenticação e autorização, faz a verificação na tabela de credenciais temporárias para saber se o Cliente possui uma credencial válida, em caso positivo ele verifica quais são os privilégios de autorização da credencial, se ele tiver permissão para acessar o serviço, sua requisição é atendida.

Caso a credencial não seja válida ou tenha expirado o Cliente é redirecionado para o servidor de Autenticação e Autorização para que possa se autenticar novamente e obter uma nova credencial conforme descrito no primeiro cenário. %Esse processo é descrito no fluxo alternativo da figura ~\ref{fig:cenario2}.

%\begin{figure}[!htb]
%    \centering
%     \includegraphics[width=0.8\textwidth]{cenario2_autenticacao.png}
%     \caption{Fluxo do protocolo de autenticação/autorização proposto, 2º cenário.}
%     \label{fig:cenario2}
%\end{figure}


\section{Formalização do protocolo}

O emprego de avaliações mais formais na área de criptografia não é recente. Grande parte dos trabalhos nesta área foram desenvolvidos na década de 90~\cite{Meadows95}. O emprego destes métodos possibilita uma análise completa do protocolo criptográfico e sua função principal é especificar se os objetivos propostos pelos autores são alcançados.

Neste trabalho, o protocolo proposto foi descrito formalmente, utilizando a lógica BAN, com o intuito de favorecer a comunicação e o entendimento utilizando uma linguagem mais precisa. Além disso, a propriedade de terminação com a geração da credencial temporária de autenticação e autorização foi verificada com um programa escrito em Prolog (ver Apêndice em anexo).


\section{Lógica BAN}

A lógica BAN foi desenvolvida por Burrows, Abadi e Needham em 1989, ela e uma das mais populares para a análise de crenças e de conhecimento entre os participantes dos protocolos criptográficos. É a primeira lógica a analisar formalmente os protocolos criptográficos,principalmente os de autenticação e distribuição de chaves~\cite{Burrows1990}.

\subsection{Notação básica}
Na lógica BAN, existem vários tipos distintos de objetos tais como entidades ou partes que se comunicam, chaves de criptografia e fórmulas lógicas. Uma fórmula lógica é uma versão idealizada da mensagem original, e às vezes ela pode ser referenciada como uma declaração lógica. Normalmente, os símbolos A, B e S denotam entidades ou participantes; $Kab$, $Kas$ e $Kbs$ denotam chaves compartilhadas; Ka, Kb e Ks denotam chaves públicas e $Ka^{-1}$, $Kb^{-1}$ e $Ks^{-1}$ denotam as chaves privadas dos participantes. Já $Na$, $Nb$ e $Ns$ são os identificadores gerados pelos participantes. As construções mais frequentemente utilizadas são apresentadas na Tabela~\ref{tab:notacaobasicaBAN}:

 \newcommand{\RHQuery}{\textbf{[???]}}
\newcommand{\RHRemark}[1]{\textbf{[#1]}}
\newcommand{\Believess}[2]{{#1}\mathrel{\textbf{\mid\equiv}}{#2}}
\newcommand{\Seess}[2]{{#1}\mathrel{\textbf{\triangleleft}}{#2}}
\newcommand{\Saids}[2]{{#1}\mathrel{\textbf{\mid\sim}}{#2}}

\newcommand{\Believes}[2]{{#1}\mathrel{\textbf{acredita}}{#2}}
\newcommand{\Sees}[2]{{#1}\mathrel{\textbf{recebeu}}{#2}}
\newcommand{\Said}[2]{{#1}\mathrel{\textbf{disse}}{#2}}
\newcommand{\Controls}[2]{{#1}\mathrel{\textbf{controla}}{#2}}
\newcommand{\Fresh}[1]{{#1}\,\textbf{novo}}
\newcommand{\Share}[3]{{#1}\stackrel{#2}{\longleftrightarrow}{#3}}
\newcommand{\ShareSecret}[3]{{#1}\stackrel{#2}{\rightleftharpoons}{#3}}
\newcommand{\PubKey}[2]{{}\stackrel{#1}{\mapsto}{#2}}
\newcommand{\Secret}[3]{{#1}\stackrel{#2}{\leftrightharpoons}{#3}}
\newcommand{\Encrypt}[2]{\{\,{#1}\,\}_{#2}}
\newcommand{\EncryptFrom}[3]{\{\,{#1}\,\}_{#2}^{#3}}
\newcommand{\Attach}[2]{\langle {#1}\rangle_{#2}}

%  \begin{tabular}{cp{16cm}}
\begin{table}[h]
    \begin{tabular}{|l|p{10cm}|}
    \hline
    \textbf{\emph{Expressão }}         & \textbf{\emph{Leitura/Significado}}                                                                               \\ \hline
    ${P}\mid\equiv{X}$               & $\Believes{P}{X}$: $P$ crê em $X$, ou $P$ acredita que $X$ é verdadeiro \\ \hline
    ${P}\triangleleft{X}$            & $\Sees{P}{X}$:Alguém enviou uma mensagem para $P$ contendo $X$ ou de outra forma $P$ recebeu $X$   \\ \hline
    ${P}\mid\sim{X}$                 & $\Said{P}{X}$:$P$ disse uma vez $X$. A entidade $P$ em algum momento enviou uma mensagem incluíndo a declaração $X$.\\ \hline
    ${P}\Rightarrow {X}$                 & $\Controls{P}{X}$:$P$ tem jurisdição sobre $X$. Onde $P$ é uma autoridade sobre $X$ é deve ser confiável \\ \hline
    \#(X)                            & novo$(X)$: a fórmula novo $X$, ela não foi usada numa mensagem anterior à execução protocolo atual.  \\ \hline
    $\Share{P}{k}{Q}$                & (lê-se ``k é uma chave satisfatória para $P$ e $Q$''). A chave $k$ nunca será descoberta por qualquer participante, exceto por P, Q ou por alguém em quem eles confiam \\ \hline
    $\{{X}\}_K$                  & fórmula $X$ foi cifrada com a chave $K$. As mensagens cifradas somente são legíveis e verificáveis pelo possuidor da chave \\ \hline
    $\ShareSecret{P}{k}{Q}$         & ${k}$ é o segredo compartilhado entre ${P}$ e ${Q}$ e possivelmente também com as entidades de confiança deles. Somente ${P}$ e ${Q}$ podem usar k para provar suas identidades \\ \hline
    $\PubKey{K}{P}$:               & ${k}$ é a chave pública de ${P}$  \\ \hline
    \end{tabular}
    \caption {Notação básica Lógica BAN, adaptação de ~\cite{Burrows1990}.}
\label{tab:notacaobasicaBAN}
\end{table}



\subsection{Postulados lógicos}

No estudo de protocolos de segurança é importante diferenciar o tempo das demonstrações ou eventos.  Haja vista que se isso não for observado, problemas, como a não detecção  do  reenvio de mensagens,  poderá acontecer.  Segundo~\cite{Burrows1990}, a lógica BAN trata dessa distinção dividindo-a em duas épocas: presente, que é o tempo durante a execução do protocolo, e o passado,  que refere-se às mensagens enviadas antes da execução do protocolo, o que faz com que elas sejam rejeitadas, uma vez que não são confiáveis .  Essa divisão de tempo é suficiente para facilitar o entendimento de como a lógica pode ser manipulada.
Para realizar a análise dos protocolos de segurança a lógica BAN possui uma serie de postulados lógicos, regras~\cite{Burrows1990}. Alguns deles  são descritos  a seguir:

\begin{enumerate}[ A )]

 \item Regra de significado da mensagem. Esta regra faz parte da interpretação das mensagens
    A1.1. Para as chaves secretas compartilhadas:
    \begin{displaymath}
        \infer
        {{P}\mid\equiv{{Q}\mid\sim{X}}}
        {{P}\mid\equiv{\Share{P}{k}{Q,}}& {P}\triangleleft{\Encrypt{X}{k}}}
        %{\Believes{P}{\Said{Q}{X}}}
        %{\Believes{P}{\Share{P}{k}{Q,}}&\Sees{P}{\Encrypt{X}{k}}}
    \end{displaymath}

    Se $P$ acredita que $k$ é uma chave satisfatória para se comunicar com $Q$ e se $P$ recebeu a mensagem $X$ cifrada com a chave $k$, então $P$ acredita que $Q$ uma vez disse $X$

    A1.2. De forma similar, aplicada a para as chaves públicas:

    \begin{displaymath}
        \infer
        {{P}\mid\equiv{{Q}\mid\sim{X}}}
        {{P}\mid\equiv{\PubKey{K}{Q}}& {P}\triangleleft{\Encrypt{X}{k ^{-1}}}}
                %{\Believes{P}{\Said{Q}{X}}}
                %{\Believes{P}{\Share{P}{k}{Q,}}&\Sees{P}{\Encrypt{X}{k}}}
    \end{displaymath}



\item Regra de verificação do identificador. Essa regra verifica se a mensagem é recente, se foi enviada durante a execução atual do protocolo e consequentemente, se o emissor acredita nela.

  \begin{displaymath}
    \infer
    {{P}\mid\equiv{{Q}\mid\equiv{X}}}
    {{P}\mid\equiv{\#(X),}& {P}\mid\equiv{{Q}\mid\sim{X}}}
    %{\Believes{P}{\Believes{Q}{X}}}
    %{\Believes{P} novo$(X),$ &\Believes{P}{\Said{Q}{X}}}
  \end{displaymath}

   Se $P$ acredita que $X$ é novo e $P$ acredita que em algum momento $Q$ disse $X$,então $P$ também acredita que $Q$ acredita em X.

\item Regra da jurisdição. Esta regra representa a confiança e a autoridade de uma entidade sobre as declarações.
\begin{displaymath}
    \infer
    {{P}\mid\equiv{X}}
    {{P}\mid\equiv{{Q}\Rightarrow {X},}& {P}\mid\equiv{{Q}\mid\equiv{X}}}
   % {\Believes{P}{X}}
   % {\Believes{P}{\Controls{Q}{X},} &\Believes{P}{\Believes{Q}{X}}}
  \end{displaymath}

   Se $P$  acredita que $Q$ tem jurisdição sobre a declaração $X$ e $P$ acredita que $Q$ acredita em $X$, então $P$ confia na declaração $X$.

\end{enumerate}

Estes são alguns dos principais postulados utilizados na construção da análise formal de protocolos criptográficos. A utilização destas regras juntamente com as notações descritas na sessão anterior possibilita que a crença dos participantes de um protocolo possa ser declarada.

\section{Análise formal do protocolo proposto}
Nesta sessão será realizado a análise formal do protocolo de autenticação e autorização proposto, utilizando a lógica BAN.

Na lógica BAN a análise de um protocolo é dividida em quatro etapas: A idealização do protocolo, que é originada a partir do protocolo original. As hipóteses, que são a suposições a respeito do estado inicial, nessa etapa são definidas as fórmulas lógicas que estão ligadas às declarações do protocolo, as declarações que são afirmações sobre o estado do sistema. As regras de inferência ou postulados lógicos que são aplicadas para os pressupostos e as afirmações a fim de descobrir as crenças detidas pelas partes no protocolo. %Normalmente, as premissas incluem as declarações sobre bens essenciais e partilha, geração de uso único e de confiança entre os diretores. nalanisisisisii %%http:www.tml.tkk.fi/Opinnot/Tik-110.501/1995/ban.html

\subsection{Idealização do protocolo}
\newcommand{\HT}[3]{\{\,{#1}\,\}\,{#2}\,\{\,{#3}\,\}}
\newcommand{\Msg}[3]{{#1}\longrightarrow{#2}:\,{#3}}

Para especificar o protocolo formalmente foram utilizadas algumas notações para representar os elementos participantes. Logo, os símbolos ${A}$, ${B}$ e ${C}$ são utilizadas para representar respectivamente as entidades, elementos participantes, que trocam mensagens: Cliente, Servidor REST e Servidor de Autenticação e Autorização. As chaves públicas das entidades ${A}$, ${B}$ e ${C}$ são representadas respectivamente por ${Ka}$, ${Kb}$ e  ${Kc}$. Já as chaves privadas, seguindo o mesmo pressuposto, são representadas pelos símbolos ${{Ka} ^{-1}}$, ${{Kb} ^{-1}}$ e ${{Kc} ^{-1}}$. Os elementos ${Cred_A}$ e ${Cod_{Srv_A}}$ representam respectivamente a credencial utilizada pela entidade ${A}$ e o código que identifica o serviço que a entidade ${A}$ está requerendo.

O desafio gerado pela entidade ${C}$ é enviado a entidade ${A}$ e representado pela fórmula ${N_{CA}}$ que corresponde aos elementos: Código do Desafio gerado pela entidade ${C}$ e Código da Credencial aleatória da entidade ${A}$. A resposta do desafio gerada pela entidade ${C}$ à entidade ${A}$ é representado  pelo símbolo ${Resp_{AC}}$ que corresponde aos elementos: Código do Desafio gerado pela entidade ${C}$ e Credencial Solicitada da entidade ${A}$. ${Ts_A}$ e ${Ts_C}$ são respectivamente os \emph{timestamps} emitidos pelas entidades ${A}$ e ${C}$.

O símbolo ${Msg_{AC}}$ representa o resumo da mensagem enviada pela entidade ${A}$ a entidade ${C}$,  ${Msg_{CA}}$ representa o resumo da mensagem enviada pela entidade ${C}$ a entidade ${A}$ e ${Msg_{AB}}$ representa o resumo da mensagem enviada pela entidade ${A}$ a entidade ${B}$. O elemento ${H}$ representa o \emph{hashing} aplicado a uma mensagem utilizando o algoritmo SHA3.

E finalmente, o símbolo ${Exp_A}$ representa a data/hora  de expiração da Credencial Temporária de autorização e autenticação, ${C_{Aut}}$ corresponde à credencial temporária de autorização e autenticação gerada para a entidade A e ${Req_A}$, que refere-se à requisição de serviço de A ( Get, Put, Post ou Delete).

Para especificar formalmente um protocolo de segurança utilizando a lógica BAN, é necessário primeiro idealizar o protocolo, objeto da análise, e a partir da aplicação dos postulados e das suposições iniciais verificar se ele atinge ou não o seu objetivo. A idealização do protocolo proposto e descrito na figura ~\ref{fig:protocoidealizado} e representa o fluxo de troca de mensagens executado pelo protocolo. Todas as mensagens são consideradas na análise, pois utilizam criptografia assimétrica desde a primeira troca de mensagens.

\begin{figure}[!htb]
    \centering
    \includegraphics[width=0.8\textwidth]{fluxo_autenticacao_BAN.png}
    \caption{Diagrama de idealização do protocolo de autenticação/autorização proposto}
    \label{fig:protocoidealizado}
\end{figure}

\begin{enumerate}
  \item Mensagem 1: $\Msg{A}{C}{\Encrypt{Ts_A,Cred_A,H{\Encrypt{Msg_{AC}}{Ka ^{-1}}}}{Kc}}$.
  \item Mensagem 2: $\Msg{C}{A}{\Encrypt{Ts_C,{N_{CA}},H{\Encrypt{Msg_{CA}}{Kc ^{-1}}}}{Ka}}$.
  \item Mensagem 3: $\Msg{A}{C}{\Encrypt{Ts_A,{Resp_{AC}},Cod_{Srv_A},H{\Encrypt{Msg_{AC}}{Ka ^{-1}}}}{Kc}}$.
  \item Mensagem 4: $\Msg{C}{A}{\Encrypt{Ts_C,Exp_A,\#(\ShareSecret{A}{{C_{Aut}}}{C}),H{\Encrypt{Msg_{CA}}{Kc ^{-1}}}}{Ka}}$.
  \item Mensagem 5: $\Msg{A}{B}{\Encrypt{Ts_A,(\ShareSecret{A}{{C_{Aut}}}{C}),H{\Encrypt{Msg_{AB}}{Ka ^{-1}}}}{Ka}},{Req_A}$.
\end{enumerate}
\subsection{Suposições}\label{sec:Suposicoes}
O objetivo do protocolo é fazer que a entidade ${A}$ seja autenticada pela entidade ${C}$ e obtenha uma credencial de autenticação e autorização temporária referente a uma requisição de um serviço que a entidade ${A}$ deseja consumir. De forma que a credencial de autenticação e autorização temporária  possa ser utilizada pela entidade ${A}$ no momento da requisição do serviço a entidade ${B}$ e obtenha o que deseja. Para isso, algumas suposições iniciais são estabelecidas e juntamente com a aplicação  dos postulados da lógica BAN, busca-se concluir que o protocolo alcance o objetivo proposto. Todas as suposições, apresentadas na Tabela~\ref{tab:suposicoesBAN} são baseadas em um canal seguro de comunicação SSL/TSL, onde tanto o receptor  quanto o emissor do serviço são conhecidos e autenticados usando-se certificados digitais X.509.

\begin{table}[h]
\begin{tabular}{cllcl}
\textbf{Suposição} & \textbf{Descrição} &  & \textbf{Suposição} & \textbf{Descrição} \\
\textbf{1 -}       & $ A \mid\equiv$  ${\PubKey{Kc}{C}}$                 &  & \textbf{9 -}       & $ B \mid\equiv$ $ C \Rightarrow $ $\#(\ShareSecret{A}{{C_{Aut}}}{C})$ \\
\textbf{2 -}       & $ B \mid\equiv$  ${\PubKey{Ka}{A}}$                 &  & \textbf{10 -}      & $ A \mid\equiv$ $ C \mid\equiv $ $\#(\ShareSecret{A}{{C_{Aut}}}{C})$ \\
\textbf{3 -}       & $ C \mid\equiv$  ${\PubKey{Ka}{A}}$                 &  & \textbf{11 -}      & $ B \mid\equiv$ $ C \mid\equiv $ $\#(\ShareSecret{A}{{C_{Aut}}}{C})$ \\

\textbf{4 -}       & $ A \mid\equiv$  $\#{Ts_C}$                         &  & \textbf{12 -}      & $ A \mid\equiv$  $\#(\ShareSecret{A}{{C_{Aut}}}{C})$  \\

\textbf{5 -}       & $ B \mid\equiv$  $\#{Ts_A}$                         &  & \textbf{13 - }      & $ B \mid\equiv$  $\#(\ShareSecret{A}{{C_{Aut}}}{C})$  \\

\textbf{6 -}       & $ C \mid\equiv$  $\#{Ts_A}$                         &  & \textbf{ }                    &                                      \\
\textbf{7 -}       & $ A \mid\equiv$  ${Exp_A}$                          &  & \textbf{ }                   &
\\
\textbf{8 -}       & $ A \mid\equiv$  $ C \Rightarrow $ $\#(\ShareSecret{A}{{C_{Aut}}}{C})$   &   & \textbf{ }     &                               \\
\end{tabular}
\caption {Suposições aplicadas ao protocolo proposto.}
\label{tab:suposicoesBAN}
\end{table}


Dessa forma, temos que as suposições 1,2 e 3 garantem que as entidades participantes ${A}$, ${B}$ e ${C}$ confiam  nas chaves públicas das entidades que farão as trocas de mensagem. As suposições 4, 5 e 6 são \emph{timestamps}, o que denota que as entidades ${A}$, ${B}$ e ${C}$ devem estar sincronizadas. Sendo assim, a entidade ${A}$ acredita que o \emph{timestamp} ${Ts_C}$ é novo e foi gerado recentemente. Da mesma forma que as entidades ${B}$ e ${C}$ acreditam  que o \emph{timestamp} ${Ts_A}$ também é novo e foi gerado recentemente. A suposição 7 é utilizada pela entidade ${A}$ para garantir que a credencial de autenticação e autorização gerado pela entidade ${C}$ não expirou e que pode ser utilizada. As suposições 8 e 9 denotam que as entidades ${A}$, ${B}$ acreditam que entidade ${C}$ tem jurisdição  sobre a credencial de autenticação e autorização gerada. Portanto, as suposições 10 e 11 garantem que as entidades ${A}$, ${B}$ acreditam que a credencial de autenticação e autorização gerada é nova é foi realmente gerada pela entidade ${C}$. Finalmente, as suposições 12 e 13 garantem que as entidades ${A}$, ${B}$ acreditam na nova credencial de autenticação e autorização temporária, ${C_{Aut}}$.

\subsection{Provas}

Como o objetivo final do protocolo é autenticar a entidade ${A}$, de forma que ela obtenha uma credencial de autenticação e autorização temporária, referente a uma requisição de serviço desejado. Será realizado uma análise de cada mensagem  do protocolo idealizado. Para isso, serão aplicados os postulados lógicos e suposições, a fim de provar que o protocolo consegue atingir o objetivo proposto.

Na primeira mensagem, a entidade ${A}$ envia sua credencial, um \emph{timestamp} e o código do serviço que ele está querendo consumir ao servidor de autenticação e autorização, representado pela entidade ${C}$. A mensagem enviada é assinada com a chave privada do participante ${A}$  e cifrada com a chave pública do participante ${C}$. Esse processo e descrito a seguir:

\textbf{Mensagem 1}: $\Msg{A}{C}{\Encrypt{Ts_A,Cred_A,H{\Encrypt{Msg_{AC}}{Ka ^{-1}}}}{Kc}}$.

$C\triangleleft$ ${\Encrypt{Ts_A,Cred_A,Cod_{Srv_A},H{\Encrypt{Msg_{AC}}{Ka ^{-1}}}}{Kc}}$

$C\mid\equiv A \mid\sim $  $H\{Msg_{AC}\}$

$C\mid\equiv A \mid\sim$ ${\#Ts_A}$

$C\mid\equiv$ ${Cred_A,Cod_{Srv_A}}$

Sendo assim, ${\textbf{C}}$ recebe a fórmula ${\Encrypt{Ts_A,Cred_A,Cod_{Srv_A},H{\Encrypt{Msg_{AC}}{Ka ^{-1}}}}{Kc}}$, e usando sua chave privada, decifra a fórmula recebida. Após decifrar a fórmula, e aplicando a regra do significado da mensagem na suposição 3 usando a função $H \{Msg_{AC}\}$ confirma a autenticidade e integridade da mensagem. Por fim, aplica a regra de verificação do identificador na suposição 6 usando a fórmula ${Ts_A}$ para obter a credencial da entidade ${A}$ :  ${Cred_A}$.

Resultado:

${C}$ obtém  a credencial da entidade $A$: ${Cred_A}$

Na segunda mensagem, após receber e validar os dados enviados por $A$, a entidade ${C}$ gera um desafio de autenticação, ${N_{CA}}$. O desafio consiste em fazer uma busca aleatória à tabela de credenciais e selecionar um código de credencial que esteja associado a entidade $A$. Em seguida grava-se o desafio, a data e hora de geração do desafio e a resposta esperada em uma base de dados. Na sequência, uma mensagem, contendo o desafio e um \emph{timestamp}, é enviada a entidade $A$, assinada com a chave privada da entidade ${C}$ e cifrada com a chave pública da entidade ${A}$. Logo temos:

\textbf{Mensagem 2}: $\Msg{C}{A}{\Encrypt{Ts_C,{N_{CA}},H{\Encrypt{Msg_{CA}}{Kc ^{-1}}}}{Ka}}$.

$\textbf{A}\triangleleft$ ${\Encrypt{Ts_C,{N_{CA}},H{\Encrypt{Msg_{CA}}{Kc ^{-1}}}}{Ka}}$

$\textbf{A}\mid\equiv \textbf{C} \mid\sim $  $H \{Msg_{CA}\}$

$\textbf{A}\mid\equiv \textbf{C} \mid\sim$ ${Ts_C}$

$\textbf{A}\mid\equiv$ ${N_{CA}}$

A entidade ${\textbf{A}}$ recebe a fórmula ${\Encrypt{Ts_C,{N_{CA}},H{\Encrypt{Msg_{CA}}{Kc ^{-1}}}}{Ka}}$ e a decifra usando sua chave privada, em seguida aplicando a regra do significado da mensagem na suposição 1 usando a função $H \{Msg_{CA}\}$ confirma a autenticidade e integridade da mensagem. Por fim, aplica a regra de verificação do identificador na suposição 4 e usando a fórmula ${Ts_C}$ para obter o desafio,${N_{CA}}$, gerado pela entidade ${C}$.

Resultado:

${A}$ obtém  o desafio de autenticação gerado pela entidade ${C}$: ${N_{CA}}$

Na terceira mensagem, a entidade ${A}$, após receber e validar o desafio gerado pela entidade ${C}$, envia a resposta ${Resp_{AC}}$, conforme solicitado. Essa resposta consiste em informar o código do desafio gerado pela entidade ${C}$ e a credencial associada ao código de credencial solicitada pela entidade ${C}$. Além disso, a entidade ${A}$ deve informar o código do serviço que deseja consumir. Sendo assim, temos:

\textbf{Mensagem 3}: $\Msg{A}{C}{\Encrypt{Ts_A,{Resp_{AC}},Cod_{Srv_A},H{\Encrypt{Msg_{AC}}{Ka ^{-1}}}}{Kc}}$.

$C\triangleleft$ ${\Encrypt{Ts_A,{Resp_{AC}},H{\Encrypt{Msg_{AC}}{Ka ^{-1}}}}{Kc}}$

$C\mid\equiv A \mid\sim $  $H \{Msg_{AC}\}$

$C\mid\equiv A \mid\sim$ ${\#Ts_A}$

$C\mid\equiv$ ${Resp_{AC}}$

${\textbf{C}}$ recebe a fórmula ${\Encrypt{Ts_A,{Resp_{AC}},H{\Encrypt{Msg_{AC}}{Ka ^{-1}}}}{Kc}}$, e a decifra usando sua chave privada, em seguida aplicando a regra do significado da mensagem na suposição 3 usando a função $H \{Msg_{AC}\}$ confirma a autenticidade e integridade da mensagem. Por fim, aplica a regra de verificação do identificador na suposição 6 usando a fórmula ${Ts_A}$ para obter os dados da resposta do desafio ${Resp_{AC}}$, e assim, validar a resposta enviada e autenticar a entidade ${A}$. Dando início ao processo de autorização.

Resultado:

${C}$ autentica a entidade ${A}$.

Na mensagem 4, após autenticar a entidade ${A}$, a entidade ${C}$ procede com o processo de autorização, que consiste em verificar qual serviço a entidade está querendo consumir, para isso verifica o código de serviço solicitado pela entidade ${A}$, $Cod_{Srv_A}$, que foi informado no envio da mensagem 3. Se a entidade ${A}$ possuir privilégios necessários para consumir o serviço requisitado, a entidade ${C}$ gera uma nova credencial de autorização temporária para o serviço solicitado. Em seguida, a entidade ${C}$ grava a credencial ${(\ShareSecret{A}{{C_{Aut}}}{C})}$, o código do serviço que a entidade ${A}$ está requerendo, a data e hora de criação, a data de expiração e o código do contrato da entidade ${A}$. Terminado esse procedimento, a entidade ${C}$, envia, uma mensagem contendo a credencial ${(\ShareSecret{A}{{C_{Aut}}}{C})}$, a data e hora de expiração da credencial,${Exp_A}$ e um \emph{timestamp} a entidade ${A}$. Esta mensagem é assinada com a privada da entidade ${C}$ e cifrada com a chave pública da entidade ${A}$. Logo temos:

\textbf{Mensagem 4}: $\Msg{C}{A}{\Encrypt{Ts_C,Exp_A,\#(\ShareSecret{A}{{C_{Aut}}}{C}),H{\Encrypt{Msg_{CA}}{Kc ^{-1}}}}{Ka}}$.

$\textbf{A}\triangleleft$ $\Msg{C}{A}{\Encrypt{Ts_C,Exp_A,\#(\ShareSecret{A}{{C_{Aut}}}{C}),H{\Encrypt{Msg_{CA}}{Kc ^{-1}}}}{Ka}}$.

$\textbf{A}\mid\equiv \textbf{C} \mid\sim $  $H \{Msg_{CA}\}$

$\textbf{A}\mid\equiv \textbf{C} \mid\sim$ ${Ts_C}$

$\textbf{A}\mid\equiv \textbf{C} \mid\sim$ ${Exp_A}$

$\textbf{A}\mid\equiv \textbf{C} \Rightarrow $  ${\#(\ShareSecret{A}{{C_{Aut}}}{C})}$

$\textbf{A}\mid\equiv \textbf{C} \mid\equiv $  ${\#(\ShareSecret{A}{{C_{Aut}}}{C})}$

$\textbf{A}\mid\equiv$ ${\#(\ShareSecret{A}{{C_{Aut}}}{C})}$

A entidade ${A}$ recebe a fórmula ${\Encrypt{Ts_C,Exp_A,\#(\ShareSecret{A}{{C_{Aut}}}{C}),H{\Encrypt{Msg_{CA}}{Kc ^{-1}}}}{Ka}}$, e a decifra usando sua chave privada, em seguida aplicando a regra do significado da mensagem na suposição 1 usando a função $H \{Msg_{CA}\}$ confirma a autenticidade e integridade da mensagem. Ela aplica a regra de verificação do identificador nas suposições 4 e 7 usando as fórmulas ${Ts_C}$  e ${Exp_A}$. E, finalmente, aplicando a regra da jurisdição, nas suposições 7 e 9, obtém a credencial de autorização temporária ${\#(\ShareSecret{A}{{C_{Aut}}}{C})}$.

Resultado:

${\textbf{A}}$ obtém a credencial de autorização temporária: ${\#(\ShareSecret{A}{{C_{Aut}}}{C})}$.

Finalmente, na quinta mensagem, a entidade ${A}$ após receber e validar a mensagem enviada pela entidade ${C}$, obtendo assim a credencial de autenticação e autorização temporária, ${\#(\ShareSecret{A}{{C_{Aut}}}{C})}$, envia uma mensagem a entidade ${B}$  contendo a requisição do serviço que deseja consumir juntamente com a credencial de autorização temporária. Esta mensagem é assinada com a chave privada da entidade ${A}$ e cifrada com a chave pública da entidade ${B}$. Logo temos:

\textbf{Mensagem 5}: $\Msg{A}{B}{\Encrypt{Ts_A,(\ShareSecret{A}{{C_{Aut}}}{C}),H{\Encrypt{Msg_{AB}}{Ka ^{-1}}}}{Kb}},{Req_A}$.

$\textbf{B}\triangleleft$  ${\Encrypt{Ts_A,(\ShareSecret{A}{{C_{Aut}}}{C}),H{\Encrypt{Msg_{AB}}{Ka ^{-1}}}}{Kb}},{Req_A}$

$\textbf{B}\mid\equiv \textbf{A} \mid\sim $  $H\{Msg_{AB}\}$

$\textbf{B}\mid\equiv \textbf{A} \mid\sim$ ${Ts_A}$

$\textbf{B}\mid\equiv \textbf{A} \Rightarrow $  ${\#(\ShareSecret{A}{{C_{Aut}}}{C})}$

$\textbf{B}\mid\equiv \textbf{A} \mid\equiv $  ${\#(\ShareSecret{A}{{C_{Aut}}}{C})}$

$\textbf{B}\mid\equiv$ ${\#(\ShareSecret{A}{{C_{Aut}}}{C})}$

A entidade ${\textbf{B}}$ recebe a fórmula ${\Encrypt{Ts_A,(\ShareSecret{A}{{C_{Aut}}}{C}),H{\Encrypt{Msg_{AB}}{Ka ^{-1}}}}{Kb}},{Req_A}$, e a decifra, usando sua chave privada. Em seguida aplicando a regra do significado da mensagem na suposição 2 usando a função $H\{Msg_{AB}\}$ confirma a autenticidade e integridade da mensagem. Ela aplica a regra de verificação do identificador na suposições 5 usando as fórmulas ${Ts_A}$. E, finalmente, aplicando a regra da jurisdição, nas suposições 8 e 10, obtém a credencial de autorização temporária ${\#(\ShareSecret{A}{{C_{Aut}}}{C})}$ e autoriza a entidade \textbf{A} a consumir a requisição ${Req_A}$.

Conclusão :

${B}$ autoriza ${A}$ a partir da nova credencial ${(\ShareSecret{A}{{C_{Aut}}}{C})}$ a consumir a requisição ${Req_A}$

\subsection{Análise}

A análise demonstra que o protocolo de autenticação e autorização proposto alcança os objetivos propostos na Seção~\ref{sec:Suposicoes}, que é a autenticação da entidade ${A}$ e a emissão da credencial de autenticação e autorização temporária. Isso permite que a entidade ${A}$ consuma o serviço requerido. É importante frisar que a lógica BAN foi utilizada para demonstrar a execução do protocolo e que a utilização da lógica é apenas um dos passos necessários para verificar se existem possíveis falhas no protocolo.

Para verificar a segurança do protocolo é necessário que seja empregado um método de criptoanálise sobre a criptografia utilizada e assim verificar quais são vulnerabilidades que o protocolo ou os algoritmos criptográficos empregados estão sujeitos.


\section{Implementação}\label{sec:implementacao}

Após a definição do protocolo proposto e para atender a necessidade da realização de testes de desempenho que tem a finalidade de avalizar o impacto da solução proposta no oferecimento de serviços pela PCDF, surgiu a necessidade da criação de um protótipo do protocolo de autenticação e autorização proposto. Esse protótipo foi desenvolvido em conjunto com o corpo acadêmico da UNB, neste caso, o responsável pelo desenvolvimento foi o aluno Alexandre Lucchesi.  A descrição desse protótipo e apresentado a seguir.
\subsection{Protótipo}

Para implementar o protótipo do protocolo proposto foi necessário desenvolver cada um dos componentes: Cliente, Servidor de Autenticação e Autorização e Servidor REST, descritos na arquitetura do protocolo na seção~\ref{sec:ArqProtocolo}. Para isso, foi utilizada a linguagem de programação puramente funcional Haskell, com o compilador GHC(Glasgow Haskell Compiler) versão 7.6.3. Para o controle de dependências e gerenciamento de \emph{build} de aplicações foi utilizado à ferramenta Cabal que é um o gerenciador de pacotes do Haskell. Com ele é possível construir aplicações e bibliotecas de forma padronizada, organizada e portável. Para criar o protótipo ainda foi utilizado o framework Haskell para desenvolvimento web, denominado Snap \emph{Framework} que é necessário para manipulação de requisições e respostas HTTP, que são utilizadas pelo protocolo de autenticação e autorização proposto. Além disso, foi utilizado o HsOpenSSL que é um OpenSSL vinculativo para Haskell, utilizado para garantir a utilização do HTTPS em todas as trocas de mensagem a partir de código Haskell.

Para a assinatura digital das mensagens utilizadas pelo protocolo foi utilizado o algoritmo RSASSA\-PKCS\-v1\_5 SHA\-256 conforme orientação descrita na publicação~\cite{ietfjws} e para criptografia foi utilizado o algoritmo de criptografia assimétrica, RSAES\-PKCS1\-V1\_5, para cifrar a chave simétrica utilizada no algoritmo de criptografia simétrica AES\_128\_CBC\_HMAC\_SHA\_256 que foi utilizado para cifrar a mensagem. Procurou-se seguir a orientação do que é preconizado na publicação~\cite{jwt2014}.

O servidor de banco de dados foi implementado utilizando o banco de dados não relacional \emph{Apache CouchDB}, que é um banco de dados flexível e tolerante a falhas que usa \emph{JSON} para armazenar os dados, JavaScript como sua linguagem de consulta usando o MapReduce, além disso, este banco de dados oferece API estilo REST.

O protótipo implementado não objetivou a otimização do protocolo, seu objetivo foi o de possibilitar a realização dos testes de desempenho e de verificar as funcionalidades do protocolo proposto.

\section{Análise de Segurança}

Nesta seção serão discutidas as propriedades de segurança do protocolo de autenticação e autorização proposto. São abordadas as propriedades e alguns ataques que podem ser realizados contra ele. Essa seção segue o padrão determinado no trabalho~\cite{traust08}.

\subsection{Segurança da sessão}

O protocolo de Autenticação e Autorização proposto utiliza para segurança de seção e da camada de transporte, o protocolo TLS/SSL. A utilização desse protocolo tem por objetivo evitar o ataque man-in-the-middle. Para isso, é exigido tanto do órgão conveniado como da própria PCDF que ambas utilizem certificados digitais padrão X.509 emitidos e garantidos por uma AC que esteja subordinada à hierarquia da ICP-Brasil.

Com isso, ambas as partes envolvidas no processo de comunicação podem estabelecer um processo de confiança mútua no nível de transporte. Todo o tráfego que flui sobre uma sessão bilateral certificada tem uma fonte confiável. Isso permite que implementações de serviços possam autorizar ou desautorizar interações com base na fonte bem conhecida de uma solicitação HTTP.

Além da segurança oferecida pela utilização da segurança na camada de transporte, com utilização do TLS/SSL, o protocolo utiliza mecanismos de segurança tais como criptografia assimétrica e assinatura digital. Isso Permite que outras partes mal intencionadas não consigam ter acesso ao conteúdo das mensagens trocadas pelo protocolo no processo de comunicação.

Diferentemente do sistema Traust, o protocolo de Autenticação e Autorização proposto, não será executado em ambientes em que nem o cliente nem o servidor tem uma chave pública certificada, procura-se dessa forma, atenuar problemas relacionados ao protocolo TLS, que quando executado em ambientes em que nem o cliente nem o servidor tem uma chave pública certificada, pode ser vulnerável a um ataque man-in-the-middle durante o estabelecimento da sessão~\cite{traust08}.

\subsection{Responsabilização dos usuários conveniados}

Uma das ameaças verificadas diz respeito a possibilidade do uso indevido, por parte dos órgãos conveniados, das informações disponibilizadas nos serviços ofertados pela PCDF. Isso decorre do fato das informações disponibilizadas serem sensíveis e possuírem caráter sigiloso. Dessa forma, para evitar esse problema é exigido dos órgãos conveniados que assinem um contrato para o consumo do serviço. No momento da assinatura do contrato eles recebem uma tabela contendo várias credenciais que servem como identidades e que deverão ser utilizadas no processo de autenticação, conforme descrito no seção~\ref{sec:reqprotocolo}. Isso possibilita que haja uma autenticação mútua entre a PCDF e os consumidores dos serviços. Além disso, com a utilização do protocolo de autenticação e autorização proposto, são empregados mecanismos de criptografia e assinaturas digitais que são geradas a partir de certificados digitais padrão X.509 vinculados aos órgãos e garantidos por AC. Dessa forma, busca-se evitar o não-repúdio por parte dos órgãos conveniados, atribuindo-lhes responsabilidades em caso do mau uso dos serviço ofertados pela PCDF.

Logo, uma vez detectado algum tipo de vazamento de dados proveniente dos serviços ofertados o órgão poderá ser identificado e após uma apuração minuciosa, responsabilizado pelos danos causados à PCDF.

\subsection{Ataques de repetição}
Esta ameaça, conforme descrito no capitulo~\ref{cap:revisaolit}, seção~\ref{sec:vulnerabilidadessoa}, se empregada contra o protocolo de autenticação e autorização proposto, tem uma chance quase nula de sucesso. Uma vez que são empregados \emph{timestamps} em em todas as trocas de mensagens realizadas pelo protocolo. Além disso, são gerados números únicos, que identificam os desafios de autenticação gerados e podem ser utilizados como \emph{nonces}, que também podem ser empregados contra esse tipo de ataque.

\subsection{Ataques de negação de serviço}

Esta ameaça é muito difícil de ser evitada, haja vista que,  pode ser fruto de ataques via rede, do consumo excessivo de recursos da máquina,  ou ainda,  ser resultante da exploração de qualquer tipo de vulnerabilidade que implique na indisponibilidade do serviço ou de um recurso. O protocolo de autenticação e autorização proposto é baseado no esquema de desafio-resposta. Os desafios são gerados de forma aleatória a partir das credenciais que identificam unicamente os consumidores dos serviços, o que possibilita uma autenticação mútua. Além disso, também são empregados outros mecanismos de segurança como utilização da segurança na camada de transporte e mecanismos de criptografia e assinaturas digitais. Isso minimiza a ameaça de ataques de negação de serviço.

Assim como o sistema Traust,~\cite{traust08}, o servidor de autenticação e autorização proposto pode ser replicado para outros servidores, o que possibilita um balanceamento de carga e minimiza a criação de gargalos. Isso permite que o servidor de autenticação seja escalável e esteja disponível em situações críticas, como no caso dos ataques de negação de serviço.


\subsection{Roubo de credenciais de autenticação e autorização}\label{subsec:RouboCred}

O protocolo de autenticação a autorização proposto após executado corretamente emite um token de segurança que será utilizado para a obtenção do serviço desejado, conforme apresenato na seção~\ref{sec:ArqProtocolo}, cabe ressaltar que se por algum motivo um atacante, conseguir burlar o os mecanismos de segurança utilizados pelo protocolo e conseguir acesso a essa credencial de autenticação e autorização ele terá acesso apenas um serviço, pois o protocolo emite credenciais para um único serviço por vez e com tempo de expiração determinado no momento de sua geração.

Essa é uma situação muito difícil de ser verificada, porém caso ocorra o problema é minimizado e não terá grande impacto na arquitetura do protocolo como um todo.

Outro problema que pode ocorrer é o roubo de credenciais de autenticação, que são as credenciais que o órgão conveniado recebe no momento da assinatura do contrato de oferecimento do serviço. Essa credenciais são utilizadas para responder os desafios que serão realizadas pelo protocolo de autenticação proposto. Caso isso ocorra e seja identificado a PCDF pode de forma rápida e transparente desativar as credenciais que julgar que foram comprometidas sem que o usuário seja prejudicado. Em um caso mais extremo, outras podem ser geradas e redistribuíras ao órgão conveniado.

Porém cabe ressaltar que caso ocorra o roubo de credenciais, o órgão que teve o problema será investigado e poderá ser responsabilizado se for detectado má fé má gestão na guarda das credenciais, conforme descrito na subseção~\ref{subsec:RouboCred}


\subsection{Ponto único de ataque}

Uma das possibilidades verificadas é que ocorrendo um ataque, o alvo possa não ser protocolo de autenticação e autorização proposto e sim o próprio servidor de autenticação e autorização. Neste caso, se o atacante for bem sucedido em sua empreitada o servidor poder ficar vulnerável. Esse servidor é responsável pela geração dos desafios de autenticação e pela geração das credenciais de autenticação e autorização. Porém, apesar de realizar estas atividades ele armazena e consulta os dados gerados no processo de autenticação e autorização em outra máquina, que é em um servidor de banco de dados. Em outras palavras o servidor de autenticação e autorização está em uma máquina diferente do servidor de banco de dados, o que minimiza os problemas relacionados a esse tipo de ataque, pois o servidor pode ser replicado para outra máquina e o que estiver com problemas pode ser fácilmente substituído.
\section{Síntese do capítulo}

Este capítulo apresentou os requisitos e a arquitetura do protocolo de autenticação e autorização proposto.  Além disso,  o capítulo apresentou a formalização do protocolo utilizando à lógica BAN.  Foi descrita de forma sucinta a implementação de um protótipo do protocolo proposto. Ao final foi realizada uma análise de segurança semelhante ao do trabalho proposto em~\cite{traust08}. No próximo capítulo será realizada uma análise de desempenho a fim de mensurar o impacto do protocolo na infraestrutura da PCDF. 
  %---------- Quinto Capítulo ----------
\chapter{Análise de Segurança e Desempenho}

Com a definição da arquitetura de referência REST e do Protocolo de Autenticação e Autorização (Capítulo~\ref{cap:Protocolo}), tornou-se necessária a realização de uma análise de segurança e de experimentos que possibilitam a mensuração do impacto da sua utilização no tratamento das requisições realizadas à PCDF.

Este capítulo inicialmente apresenta uma análise dos mecanismos de segurança propostos no protocolo. Essa primeira análise segue um procedimento informal, mas está de acordo com trabalhos descritos na literatura~\cite{traust08} e~\cite{Altair2004}. Em seguida, este capítulo apresenta uma avaliação empírica que busca analisar o desempenho do Protocolo de Autenticação e Autorização proposto.

\section{Análise de Segurança}

Nesta seção serão discutidas as propriedades de segurança do Protocolo de Autenticação e Autorização proposto. São abordadas as propriedades e alguns ataques que podem ser realizados. Essa seção segue a estrutura e se fundamenta (parcialmente) no discurso utilizado na análise do protocolo Traust~\cite{traust08}.

\subsection{Segurança da sessão}

O protocolo de Autenticação e Autorização proposto utiliza para segurança de sessão e camada de transporte o protocolo TLS/SSL. A utilização desse protocolo tem por objetivo evitar o ataque \emph{man-in-the-middle}. Para isso, é exigido, tanto para o órgão conveniado quanto para a própria PCDF, a utilização de certificados digitais padrão X.509, emitidos e garantidos por uma CA que esteja subordinada à hierarquia da ICP-Brasil. Ambas as partes envolvidas no processo de comunicação podem estabelecer um processo de confiança mútua no nível de transporte. Todo o tráfego que flui sobre uma sessão bilateral certificada tem uma fonte confiável permitindo que implementações de serviços possam autorizar ou desautorizar interações com base na fonte bem conhecida de uma solicitação HTTP.

Além da segurança oferecida pela utilização da segurança na camada de transporte, com utilização do TLS/SSL, o protocolo utiliza mecanismos de segurança tais como criptografia assimétrica e assinatura digital. Isso permite que outras partes mal intencionadas não consigam ter acesso ao conteúdo das mensagens trocadas pelo protocolo no processo de comunicação.

Diferentemente do sistema Traust~\cite{traust08}, o Protocolo de Autenticação e Autorização proposto, será executado apenas em ambientes que tanto o cliente quanto o servidor possuam chaves públicas certificadas. Procura-se, dessa forma, atenuar problemas relacionados ao protocolo TLS, que quando executado em ambientes em que as partes n\~{a}o possuem chaves públicas certificadas, relacionados a ataques \emph{man-in-the-middle} durante o estabelecimento da sessão~\cite{traust08}.

\subsection{Responsabilização dos usuários conveniados}

Uma das ameaças verificadas diz respeito a possibilidade do uso indevido, por parte dos órgãos conveniados, das informações disponibilizadas nos serviços ofertados pela PCDF. Tal fato decorre porque as informações disponibilizadas s\~{a}o sensíveis e possuem caráter sigiloso. Dessa forma, para evitar esse problema é exigido dos órgãos conveniados que assinem um contrato para o consumo do serviço. No momento da assinatura do contrato as institui\c c\~{o}es recebem uma tabela contendo várias credenciais que servem como identidade e que deverão ser utilizadas no processo de autenticação, conforme descrito no seção~\ref{sec:reqprotocolo}. Isso possibilita que ocorra uma autenticação mútua entre a PCDF e os consumidores dos serviços.

Além disso, com a utilização do Protocolo de Autenticação e Autorização proposto, são empregados mecanismos de criptografia e assinaturas digitais que são geradas a partir de certificados digitais padrão X.509 vinculados aos órgãos e garantidos por uma CA. Deste modo, busca-se evitar o não-repúdio por parte das instituições conveniadas, atribuindo-lhes responsabilidades em caso do mau uso dos serviços ofertados pela PCDF. Logo, uma vez detectado algum tipo de vazamento de dados proveniente dos serviços ofertados o órgão poderá ser identificado e após uma apuração minuciosa, responsabilizado pelos danos causados à PCDF.

\subsection{Ataques de repetição}

Esta ameaça, conforme descrita na Seção~\ref{sec:vulnerabilidadessoa}, se empregada contra o Protocolo de Autenticação e Autorização proposto, tem baixa probabilidade de sucesso, haja vista que utilizamos \emph{timestamps} na maior parte das mensagens trocadas pelo protocolo. Além disso, são gerados números únicos que identificam os desafios de autenticação gerados e que podem ser utilizados como \emph{nonces}, valores gerados de forma aleat\'{o}ria e que podem ser empregados contra esse tipo de ataque.

\subsection{Ataques de negação de serviço}

Esta ameaça é muito difícil de ser evitada, uma vez que pode ser fruto de ataques via rede, do consumo excessivo de recursos da máquina  ou ainda, ser resultante da exploração de qualquer tipo de vulnerabilidade que implique na indisponibilidade do serviço ou de um recurso.

O Protocolo de Autenticação e Autorização proposto é baseado no esquema de desafio-resposta. Os desafios são gerados de forma aleatória a partir das credenciais que identificam unicamente os consumidores dos serviços, o que possibilita uma autenticação mútua. Além disso, também são empregados outros mecanismos de segurança como utilização da segurança na camada de transporte e mecanismos de criptografia e assinaturas digitais. Isso minimiza a ameaça de ataques de negação de serviço.

Assim como o protocolo Traust~\cite{traust08}, o servidor de autenticação e autorização proposto pode ser replicado para outros servidores, o que possibilita um balanceamento de carga e minimiza a criação de gargalos, permitindo que o servidor de autenticação seja escalável e esteja disponível em situações críticas, como no caso dos ataques de negação de serviço.


\subsection{Roubo de credenciais de autenticação e autorização}\label{subsec:RouboCred}

O Protocolo de Autenticação e Autorização proposto, após executado corretamente, emite um token de segurança que será utilizado para a obtenção do serviço desejado, conforme apresentado na seção~\ref{sec:ArqProtocolo}. Cabe ressaltar que se um atacante conseguir burlar os mecanismos de segurança utilizados pelo protocolo e conseguir acesso a essa credencial de autenticação e autorização, o atacante terá acesso apenas a um serviço, pois o protocolo emite credenciais para um único serviço por vez e com tempo de expiração determinado no momento de sua geração. Essa é uma situação muito difícil de ser verificada, porém caso o extravio de credenciais ocorra, buscamos minimizar o problema com o isolamento de uma credencial por serviço requisitado, diminuindo o impacto que pode ocorrer.

Outro problema que pode ocorrer é o roubo de credenciais de autenticação, que são as credenciais que o órgão conveniado recebe no momento da assinatura do contrato de oferecimento do serviço. Essas credenciais são utilizadas para responder aos desafios que serão realizados pelo Protocolo de Autenticação e Autorização proposto. Caso o extravio ocorra e seja identificado a PCDF pode, de forma flexível e transparente, desativar as credenciais que julga terem sido comprometidas sem que o usuário seja prejudicado.
Em um caso mais extremo, outras credenciais podem ser geradas e redistribuídas \`{a}s institui\c c\~{o}es conveniadas. Porém, cabe ressaltar que, caso ocorra o extravio de credenciais, o órgão que teve o problema será investigado e poderá ser responsabilizado, caso seja detectada uma má gestão na guarda das credenciais, conforme descrito na subseção~\ref{subsec:RouboCred}


\subsection{Ponto único de ataque}

Uma das possibilidades verificadas é que, ocorrendo um ataque, o alvo possa não ser Protocolo de Autenticação e Autorização proposto e sim o próprio \servidorAA.
Neste caso, se o atacante for bem sucedido o servidor pode ficar vulnerável. Esse servidor é responsável pela geração dos desafios de autenticação e pela geração das credenciais de autenticação e autorização. Porém, apesar de realizar estas atividades, ele armazena e consulta os dados gerados no processo de autenticação e autorização em outra máquina, que é um servidor de banco de dados. Em outras palavras, o \servidorAA está em uma máquina diferente do servidor de banco de dados, o que minimiza os problemas relacionados a esse tipo de ataque, pois o servidor pode ser replicado para outra máquina e o que estiver com problemas pode ser facilmente substituído.

\subsection{S\'{i}ntese da An\'{a}lise de Seguran\c ca}

O Protocolo de Autenticação e Autorização proposto busca atender os atributos básicos de segurança, tais como: confidencialidade, integridade e autenticidade. O Protocolo incorpora mecanismos de segurança tais como: criptografia, assinatura digital e uso de certificados digitais, além da utilização do SSL/ TLS para prover segurança na camada de transporte. Ele se mostra eficiente contra vários tipos de ataques, como por exemplo: \emph{man-in-the-middle} e ataques de repetição. Contudo, alguns pontos requerem mais atenção, como por exemplo: preocupações com ataques do lado do cliente, uma vez que esses ataques são cada vez mais comuns.

\section{Testes de desempenho}

Testes de desempenho são definidos como investiga\c c\~{o}es técnicas realizadas para determinar a capacidade de resposta, a confiabilidade ou escalabilidade de um sistema, sob uma determinada carga de trabalho~\cite{Meier2007}.
A análise de desempenho possibilita identificar problemas que geralmente são encontrados em sistemas computacionais.
Esses problemas podem ser agrupados em termos de comparação e configuração de sistemas, identificação de gargalos, caracterização de cargas de trabalho e a previsão de desempenho~\cite{jain1991art}. Dessa forma, foram conduzidos experimentos com o objetivo de analisar o desempenho do Protocolo de Autenticação e Autorização proposto. O restante dessa se\c c\~{a}o descreve os procedimentos e os resultados obtidos com a an\'{a}lise de desempenho.

\subsection{Objetivos, Questões e Métricas}\label{sec:gqm}

A avaliação de desempenho do Protocolo de Autenticação e Autorização proposto foi organizada com o uso da abordagem
GQM (\emph{Goals, Questions, and Metrics})~\cite{gqm}. O resultado da aplicação da GQM é a especificação de um sistema de
medição visando um conjunto particular de problemas e um conjunto de regras para a interpretação dos dados de medição~\cite{gqm}.
O modelo de avaliação resultante tem três níveis: nível conceitual, onde  são definidos os objetivos da medição; nível operacional, que é aquele onde são verificadas as questões  que caracterizam o objeto da medição; e o nível quantitativo, que é o nível onde são definidas as métricas que identificam a medidas necessárias para responder as questões levantadas~\cite{gqm}.

\subsubsection{Objetivos }\label{sec:gqmobjetivos}

A realização de testes de desempenho objetivam determinanr se o desempenho de um algoritmo, protocolo ou sistema está de acordo com os requisitos. Outro objetivo dos testes de desempenho é o de validar a capacidade de resposta, a vazão, a confiabilidade e a escalabilidade de um sistema sob uma determinada carga~\cite{Meier2007}. Em relação ao trabalho desenvolvido nessa dissertação, a avaliação de desempenho do Protocolo de Autenticação e Autorização proposto tem como objetivo:

\begin{itemize}
\item Verificar o impacto do protocolo de autenticação e autorização proposto;
\item Identificar a viabilidade do uso do protocolo em cen\'{a}rios de carga esperados para a PCDF.
\end{itemize}


\subsubsection{Questões}\label{sec:gqmquestoes}

Seguindo o que é preconizado na metodologia GQM, os objetivos são definidos em um nível abstrato, de forma que devem ser formuladas questões, em um nível operacional, que têm por finalidade responder se os objetivos definidos serão atendidos. Logo, com base nos objetivos formulados para a análise de desempenho do Protocolo de Autenticação e Autorização proposto, as seguintes questões foram investigadas:

\parbox{0.8\textwidth}{
\begin{enumerate}[(Q1)]
\item \emph{Qual o impacto observado no tempo de resposta às requisições com o uso do protocolo?}
\item \emph{Um protótipo funcional, sem foco em otimização, consegue suportar a demanda prevista?}
\end{enumerate}}

\subsubsection{Métricas}

O desempenho é descrito quantitativamente por meio de métricas. Uma métrica pode ser definida como um conjunto de dados que é definido para responder uma questão de maneira quantitativa. Dessa forma, foram utilizadas as seguintes métricas na avaliação do desempenho do Protocolo de Autenticação e Autorização proposto.

\begin{itemize}
\item quantidade de usuários: vari\'{a}vel de controle que representa a quantidade de usuários utilizados na avaliação de desempenho.

\item utilização do protocolo: vari\'{a}vel de controle que indica a utiliza\c c\~{a}o ou n\~{a}o do protocolo proposto na disserta\c c\~{a}o. Ela é definida em dois níveis, SIM ou  NÃO.

\item tempo médio de resposta: vari\'{a}vel de resposta que consiste no tempo entre a solicitação de um serviço por um usuário até o momento em que ele recebe uma resposta
completa~\cite{ Molyneaux2009}. Para a análise de desempenho do protocolo proposto o tempo médio será dado em milissegundos.

\item Vazão (\emph{throughput}): vari\'{a}vel de resposta que corresponde ao número de operações que podem ser tratadas pelo protocolo em um
determinado período de tempo~\cite{ Molyneaux2009}.

\end{itemize}

Dessa forma, após definir as métricas utilizadas na avaliação de desempenho, foi criado um plano de medição que descreve
como as vari\'{a}veis de resposta serão mensuradas e quais procedimentos ser\~{a}o realizados durante o período de execução dos experimentos.

Logo, o primeiro passo foi a criação de um serviço REST, que retorna informações sobre ocorrências policiais, tais como:
quais tipos de ocorrências criminais estão registradas, dados gerais da vítima, autor, dentre outras informações.
Esse serviço é consumido  no momento da execução dos testes de desempenho da solução proposta.
Foram realizados 10 cen\'{a}rios de testes, conforme representado na Tabela~\ref{tb:tb_testes}.

\begin{table}[h]
\begin{center}
\begin{tabular}{|c|c|c|}
\hline
N. de usuários & Sem utilização do protocolo & Com utilização do Protocolo \\ \hline
10             & Teste 1                     & Teste 2                     \\ \hline
20             & Teste 3                     & Teste 4                    \\ \hline
30             & Teste 5                     & Teste 6                    \\ \hline
40             & Teste 7                     & Teste 8                    \\ \hline
50             & Teste 9                     & Teste 10                    \\ \hline
\end{tabular}
\caption {Cen\'{a}rios de testes realizados}\label{tb:tb_testes}
\end{center}
\end{table}

Para a execução dos testes, que utilizam as métricas definidas, foram estabelecidas as seguintes regras: Cada teste deve realizar várias requisições ao serviço REST criado e o número de requisições é obtido pela fórmula $numeroUsuarios \times 100$. A frequência de lançamento das threads, que representam os usuários virtuais, é definido pela fórmula $numeroUsuarios \times 2s$. Exemplificando,
de acordo com a Tabela~\ref{tb:tb_testes}, ao selecionar o Teste 1, serão executadas 10 threads a cada 20 segundos.
Importa salientar que cada thread corresponde a um usuário, é iniciada 2 segundos após a thread anterior e executa 100 requisições---
o que totaliza 1000 requisições ao serviço no cen\'{a}rio Teste 1. Esse procedimento foi adotado para os demais cen\'{a}rios de testes.
Por fim, as variáveis de resposta adotadas para a análise de desempenho, conforme discutido, foram o tempo médio de resposta das requisições e a vazão média de requisições por segundo.

\subsection{Configuração do ambiente de teste}

O ambiente de teste envolveu estações de trabalho tanto do Laboratório de Engenharia de Software da Universidade de Brasília quanto na
Divisão de Tecnologia da Policia Civil do Distrito Federal. Para a realização dos testes foram utilizadas configurações distintas de computadores, conforme apresentado na tabela~\ref{tb:estudo_caso1}.

\begin{table}[h]
\begin{center}
    \begin{tabular}{|p{6cm}|c|p{6cm}|}
    \hline
    Esta\c c\~{a}o                               & Quantidade & Configura\c c\~{a}o \\ \hline
    Servidor de Autenticação e Autorização       & 1          & Desktop DELL Intel Core I5-2450 2,5 GHz, 4 Gb RAM, 500 HD \\ \hline
    Servidor de Fachada REST                     & 1          & Desktop DELL Intel Core I5-2450 2,5 GHz, 4 Gb RAM, 500 HD \\ \hline
    Cliente REST                                 & 1          & Desktop DELL Intel Core I5-2450 2,5 GHz, 4 Gb RAM, 500 HD \\ \hline
    Servidor do Serviço web REST PCDF            & 1          & Intel Xeon E7 4870 2,4 GHz, 8 Gb RAM, 120 HD \\ \hline
    \end{tabular}
    \caption {Ambiente utilizado na análise de desempenho}\label{tb:estudo_caso1}
\end{center}
\end{table}

Os servidores de Autenticação e Autorização, Fachada REST e Cliente REST, foram prototipados utilizando a linguagem de programação funcional \emph{Haskell}. Estes servidores acessam uma base de dados não relacional \emph{CouchDB}, conforme apresentado na seção~\ref{sec:implementacao}.
O sistema operacional utilizado nessas máquinas foi o Linux Ubuntu 12.04 LTS-64 bits. No caso do serviço REST desenvolvido pela PCDF,
para atender a demanda da análise de desempenho, foi desenvolvido utilizando a linguagem C\# acessando um banco de dados \emph{SQL Server 2008 r2} que mant\'{e}m os registros das ocorr\^{e}ncias policiais. O serviço foi publicado em um servidor que utiliza o sistema operacional
\emph{Windows Server 2008 r2, Enterprise Edition x64}, utilizando o ISS 7.0 como servidor Web.
A Figura~\ref{fig:ambiente_teste} descreve os ambientes de configuração utilizados na análise de desempenho do Protocolo de Autenticação e Autorização proposto.


\begin{figure}[!htb]
\centering
\includegraphics[width=0.8\textwidth]{ambiente_teste_desempenho.png}
\caption{Ambiente de teste de desempenho do Protocolo de Autenticação e Autorização.}
\label{fig:ambiente_teste}
\end{figure}

\subsection{Análise dos resultados}\label{sec:analise_resultados}

Os testes objetivam: (a) verificar o impacto no tempo de resposta às requisições com o uso do protocolo e (b) avaliar se um protocolo funcional, sem foco em otimização, suportaria a demanda de requisi\c c\~{o}es prevista para a PCDF. Para atender aos objetivos supracitados, foram realizados testes considerando requisi\c c\~{o}es diretas, sem o uso do protocolo de autentica\c c\~{a}o e autoriza\c c\~{a}o; e requisi\c c\~{o}es seguras e que seguem as trocas de mensagens definidas no protocolo. Conforme discutido nas se\c c\~{o}es anteriores, as an\'{a}lises resultaram em um universo de amostras com 1000, 2000, 3000, 4000 e 5000 requisi\c c\~{o}es, correspondendo respectivamente a grupos de 10, 20, 30, 40 e 50 usuários.
Os dados coletados foram analisados, e resultados estatísticos são apresentados nas  tabelas~\ref{tb:estatistica_com_cripto} e ~\ref{tb:estatistica_sem_cripto}.

Dessa forma, com o objetivo de responder a primeira questão relacionada ao teste de performance
(\emph{Qual o impacto observado no tempo de resposta às requisições com o uso do protocolo?}),
consideramos que os resultados obtidos evidenciam um impacto significativo com a utilização do Protocolo de Autenticação e Autorização proposto.
Pode-se observar que, com a utilização do protocolo para 10 usuários simultâneos, o tempo médio de resposta é de 683,84 milissegundos.
Por outro lado, quando comparado ao acesso ao serviço sem a utilização do protocolo, o tempo médio de resposta às requisições é de 42,26 milissegundos.
Estas comparações estendem-se aos demais cen\'{a}rios de testes, para 20, 30, 40 e 50 usuários. Em todos casos o impacto é significativo quando comparado aos resultados obtidos sem a utilização do Protocolo de Autenticação e Autorização. Porém, essa variação era de certa forma esperada, pois o protocolo proposto incorpora mecanismos de segurança que requerem algoritmos de criptografia e assinatura digital e segurança na camada de transporte com a utilização do SSL/TLS. Além disso, a implementação do protocolo será ainda alvo de otimizações.

De outra forma, ao analisar o tempo médio dos experimentos que utilizaram o protocolo, nota-se que este tempo é aceitável, pois com 10 usuários ele fica abaixo de 1 segundo, com 20 o tempo médio é 1,5 segundos, com 30 usuários esse tempo sobe para aproximadamente 2,5 segundos, aumentado de forma aceitável até 50 usuários em que o tempo médio verificado foi de aproximadamente 4,5 segundos, conforme apresentado nas tabelas ~\ref{tb:estatistica_com_cripto} e~\ref{tb:estatistica_sem_cripto} e no gráfico~\ref{fig:grafico_teste_desempenho}. De forma que esses resultados estão dentro do esperado para utilização. Atualmente o número de convênios na PCDF não supera 10 usuários, estima-se que com a utilização do protocolo esse número suba para no máximo 30 órgãos conveniados. Logo, com a análise dos resultados percebe-se que, apesar do impacto da utilização do protocolo de autenticação e autorização, o tempo médio de resposta não configura como um limitador para sua utilização. %{\color{red}Relaciona com os estudos que avaliam o impacto no tempo de resposta com a ado\c c\~{a}o de WS-Security, WS-*.}


\begin{table}[h]
\begin{center}
\begin{tabular}{|c|c|c|c|c|c|}
\hline
%\multicolumn{6}{|p{15cm}|}{\textbf{\begin{tabular}[c]{@{}c@{}}   TEMPO DE RESPOSTA COM A UTILIZAÇÃO \\      DO PROTOCOLO DE AUTENTICAÇÃO E AUTORIZAÇÃO \end{tabular}}} \\ \hline
%\multicolumn{1}{|l|}{Qtd Usuários}    & Tempo Médio   & Erro Padrão & Mediana  & Desv Padrão  & Qtd Req. \\ \hline
Qtd Usuários    & Tempo Médio   & Erro Padrão & Mediana  & Desv Padrão  & Vazão    \\ \hline
10              & 683,84        & 12,53       & 622,5    & 396,36       & 11,60    \\ \hline
20              & 1.431,14      & 21,63       & 1.274,5  & 967,30       & 11,02    \\ \hline
30              & 2.466,28      & 35,86       & 2.067    & 1964,30      & 9,66     \\ \hline
40              & 3.249,76      & 35,83       & 3.084,5  & 2266,50      & 9,82     \\ \hline
50              & 4.677,64      & 44,84       & 4.375,5  & 3170,80      & 8,62     \\ \hline
\end{tabular}\caption {Análise de desempenho considerando o protocolo de autenticação e autorização.}\label{tb:estatistica_com_cripto}
\end{center}
\end{table}

\begin{table}[h]
\begin{center}
\begin{tabular}{|c|c|c|c|c|c|}
\hline
%\multicolumn{6}{|p{15cm}|}{\textbf{\begin{tabular}[c]{@{}c@{}}TEMPO DE RESPOSTA SEM UTILIZAÇÃO \\ DO PROTOCOLO DE AUTENTICAÇÃO E AUTORIZAÇÃO\end{tabular}}} \\ \hline
%\multicolumn{1}{|l|}{Qtd Usuários}    & Tempo Médio    & Erro Padrão & Mediana  & Desv Padrão & Qtd Req. \\ \hline
Qtd Usuários    & Tempo Médio    & Erro Padrão & Mediana  & Desv Padrão & Qtd Req. \\ \hline
10              & 42,26          & 1,12        & 43       & 35,51       &  1000       \\ \hline
20              & 40,96          & 0,65        & 39       & 29,35       &  2000       \\ \hline
30              & 42,79          & 1,20        & 37       & 65,97       &  3000       \\ \hline
40              & 40,60          & 0,40        & 43       & 25,69       &  4000       \\ \hline
50              & 42,64          & 0,48        & 44       & 34,39       &  5000       \\ \hline
\end{tabular}\caption {An\'{a}lise de desempenho sem considerar o protocolo de autenticação e autorização.}\label{tb:estatistica_sem_cripto}
\end{center}
\end{table}

Al\'{e}m disso, a quantidade de usuários simultâneos tem impacto no desempenho da aplicação SOA que utiliza o Protocolo de Autenticação e Autorização proposto. Verifica-se que o tempo médio de resposta sobe de acordo com a quantidade de usuários. No caso da não utilização do protocolo, observa-se que o tempo médio de resposta é constante e que a variação não \'{e} tão significante com o aumento do número de usuários simultâneos, conforme observado na figura~\ref{fig:grafico_teste_desempenho}.

\begin{figure}[!htb]
    \centering
    \includegraphics[width=0.8\textwidth]{grafico_teste_desempenho2.png}
    \caption{Comparativo de utilização do Protocolo de Autenticação e Autorização.}
    \label{fig:grafico_teste_desempenho}
\end{figure}
%análise comparativa com WS-Security%

%///////////////////


Para responder a segunda questão relacionada \`{a} an\'{a}lise de desemepenho (\emph{Um protótipo funcional, sem foco em otimização, consegue suportar a demanda prevista?}), novamente foram realizadas análises dos dados coletados nos testes realizados. A análise dos resultados obtidos com a execução do experimento ocorreu com a observação do tempo médio de resposta e com a verificação da vazão, ou seja, o número de requisições atendidas por segundo, que foi coletado ao final da execução do Teste 10, aplicado a um grupo de 50 usuários, conforme apresentado na Tabela~\ref{tb:tb_testes}. Dessa forma, foram coletadas 5000 amostras que utilizaram o Protocolo de Autenticação e Autorização proposto. Os resultados estatísticos são apresentados na tabela~\ref{tb:estatistica_com_cripto}, descrita anteriormente.

Observa-se que a PCDF, em um cenário extremo, pode ter que responder a uma média de 1200 requisições de um serviço que fornece informações sobre ocorrências policiais em um período de uma hora, ou seja, duas requisições por segundo. Neste caso, os resultados obtidos com o experimento realizado com os 50 usuários, demonstrou que o tempo médio de resposta foi de 4677 milissegundos (ou aproximadamente 4,677 segundos) e que a vazão média, que é o número de requisições atendidas por segundo, foi de 8,62 requisições por segundo. Ao se realizar a comparação deste cenário com o cenário extremo vivenciado pela PCDF, verificou-se que a arquitetura, do ponto de vista de vazão, atende ao cen\'{a}rio mais cr\'{i}tico, que é de 2 duas requisições por segundo. O que garante a qualidade do serviço ofertado pelo protocolo de autenticação e autorização proposto.

\subsection{Impacto de Mecanismos de Segurança no Tempo de Resposta}

Conforme os resultados apresentados na seção~\ref{sec:analise_resultados} o uso do Protocolo proposto com 10 usuários leva a um acréscimo no tempo resposta de uma ordem de magnitude, e para 50 usuários esse impacto é de duas ordens de magnitude. Outros estudos avaliam o impacto da introdução de mecanismos de segurança e são descritos a seguir.

O primeiro estudo verificado, é um trabalho que utiliza WS-Security para prover segurança à serviços que utilizam a tecnologia SOAP. O trabalho foi desenvolvido por Douglas Rodrigues, Julio C Estrella e Kalinka RLJC Branco~\cite{rodrigues2011analysis}. Nesse trabalho é realizado um estudo sobre segurança aplicada a arquiteturas orientadas a serviço, sendo desenvolvida uma análise de desempenho de Web Services seguros utilizando o padrão WS-Security, que é um padrão utilizado para prover segurança a Web Services.

No referido trabalho foram implementados quatro serviços que executam uma mesma operação. O que diferencia um serviço do outro é a política de segurança estabelecida. A política de segurança utilizada nesses serviços foi a do uso de algoritmos criptográficos. Os serviços disponibilizados para a análise de desempenho ficaram configurados da seguinte maneira: serviço sem política de segurança, serviço com política de segurança voltado apenas para criptografia, serviço com política de segurança voltado apenas ao uso da assinatura digital e um serviço com política de segurança que utiliza criptografia e assinatura digital.

Dessa forma, foram realizados oito experimentos que verificaram o uso ou não de criptografia, o uso ou não da assinatura digital e o número de clientes, 5 e 10 clientes, acessando o serviço de forma concorrente. A tabela~\ref{tb:expWSSecurity} representa os experimentos realizados no trabalho~\cite{rodrigues2011analysis}. A variável de resposta dos experimentos foi o tempo de resposta.

\begin{table}[h]
\begin{center}
\begin{tabular}{|c|c|c|}
\hline
\textbf{Experimento} & \textbf{Política de segurança}     & \textbf{Numero de Clientes} \\ \hline
1                    & Sem segurança                      & 5                           \\ \hline
2                    & Criptografia                       & 5                           \\ \hline
3                    & Assinatura digital                 & 5                           \\ \hline
4                    & Assinatura digital e Criptografia  & 5                           \\ \hline
5                    & Sem segurança                      & 10                           \\ \hline
6                    & Criptografia                       & 10                           \\ \hline
7                    & Assinatura digital                 & 10                           \\ \hline
8                    & Assinatura digital e Criptografia  & 10                          \\ \hline
\end{tabular}\caption {Análise de desempenho de Web Services Seguros~\cite{rodrigues2011analysis}.}\label{tb:expWSSecurity}
\end{center}
\end{table}

Para efeito de análise foram considerados apenas os experimentos que se assemelharam a análise de desempenho proposta nesta dissertação, ou seja, os experimentos que utilizam a assinatura digital e criptografia.  Sendo assim, o tempo de resposta  que referem-se aos resultados obtidos a partir do trabalho~\cite{rodrigues2011analysis} para  5 e 10 clientes, são respectivamente: 3094,20 e 6038,10 milissegundos. Esses valores quando comparados com os experimentos que não utilizaram segurança cujo tempo médio de resposta foi de 283,68 e  482,59,  para 5 e 10 usuários respectivamente, conforme apresentado na figura~\ref{fig:comparativowssecurity}. Os resultados  evidenciam que o impacto do uso do WS-Security é significativo e demonstra uma queda no desempenho dos Web services que utilizam mecanismos de segurança.


\begin{figure}[!htb]
    \centering
    \includegraphics[width=0.9\textwidth]{comparativowssecurity.png}
    \caption{Tempo médio para 5 e 10 usuários.}
    \label{fig:comparativowssecurity}
\end{figure}

Outro estudo analisado é um artigo que apresenta uma nova abordagem para fornecer segurança para serviços RESTful equivalente ao padrão WSSecurity~\cite{verstichel}. Nesse trabalho foi realizada uma avaliação de desempenho, considerando vários cenários, com a finalidade de analisar o impacto da proteção de mensagem para o desempenho de Web Services~\cite{verstichel}.

Dessa forma, foram executados diversos testes de desempenho que compararam  serviços, com mecanismos de segurança, baseados na arquitetura REST e os  Web services SOAP que utilizam o padrão WS-Security para prover segurança. Os experimentos foram realizados em um ambiente com a mesma configuração. Na análise de desempenho abordada no trabalho~\cite{verstichel}, são definidos e implementados três cenários. No primeiro cenário é realizado um consulta simples (Get) tanto para Web services RESTFul quanto Web services SOAP. No segundo cenário  segue-se a mesma abordagem, só que neste caso é realizado uma  modificação (Post) de uma mensagem e por fim, o terceiro cenário (Large), aborda o envio de uma grande quantidade de dados entre o cliente e o servidor. Para realizar a medição do experimento foram utilizadas as seguintes métricas: não uso de mecanismos de segurança(Plain),  uso da criptografia (Enc), uso da assinatura (Sign) e uso da criptografia e assinatura( Sign\&Enc). Logo, a análise compreendeu o estudo de cenários heterogêneos a fim de comparar os diferentes mecanismos de segurança~\cite{verstichel}. Os resultados são sintetizados e apresentados na figura~\ref{fig:comparativoSOAP_REST}.

\begin{figure}[!htb]
    \centering
    \includegraphics[width=1.0\textwidth]{analise_desempenho_estudo2.png}
    \caption{Comparação de tamanho de Payload e Header e Tempo Médio de processamento de mensagens SOAP e REST~\cite{verstichel}.}
    \label{fig:comparativoSOAP_REST}
\end{figure}

Sendo assim, ao analisar o tempo de resposta dos experimentos realizados, pode-se verificar que há um impacto significativo com a adoção dos mecanismos de segurança. Sejam eles voltados a Web services RESTful ou Web services SOAP, neste caso, utilizando o WS-Security. Isso fica evidenciado em todos os cenários verificados. Exemplificando, no cenário de consulta simples (Get), no caso em que é utilizado a assinatura e criptografia (Sing\&Enc) o tempo médio para Web services SOAP (que utilizam o padrão WS-Security) e Web services REST (com mecanismos de segurança) é de 6.244 e 3.156 milissegundos, aproximadamente 6,25 e 3,2 segundos. Ao realizar a comparação com o caso da não utilização de segurança (Plain) o tempo é reduzido significativamente para menos de 1 segundo.

Por fim, o último estudo analisado, foi o desenvolvido por  Pedro Miguel Gonçalves Ferreira~\cite{pedromiguel2012}, neste trabalho buscou-se identificar e caracterizar as diferentes fontes de informação da Universidade de Aveiro (UA) sendo necessário  construir APIs ou serviços web de acesso à informação. Dessa forma, entre os projetos desenvolvidos, foi implementado um projeto denominado myPersonas, que é um sistema que permite a gestão por parte do utilizador, de várias identidades num cenário de multi-provedores de serviços~\cite{pedromiguel2012}. Neste caso, foi utilizado o padrão WS-Security para prover segurança aos serviços ofertados pelo projeto.

Logo, após desenvolver o projeto e com a finalidade de avaliar o desempenho da solução implementada, foram executados 4 (quatro) testes. No primeiro teste, o serviço web descrito foi executado sem qualquer tipo de mecanismo de segurança. No segundo teste foi implementado apenas a assinatura do XML. No terceiro teste foi implementada apenas a criptografia das mensagens SOAP. No quarto e último teste foram implementadas tanto criptografia como a assinatura da mensagem SOAP. Nos experimentos que utilizaram mecanismos de segurança o padrão adotado para prover segurança ao Web service foi o WS-Security. Para todos os experimentos foi utilizado um Web service SOAP simples, que efetua a soma entre dois números. Os resultados dos experimentos são apresentados no gráfico~\ref{fig:tempowssecurity}.


\begin{figure}[!htb]
    \centering
    \includegraphics[width=0.8\textwidth]{tempo_ws_security.png}
    \caption{Tempos de execução dos testes de desempenho com e sem a utilização do WS-Security, adaptado de~\cite{pedromiguel2012}.}
    \label{fig:tempowssecurity}
\end{figure}

Ao analisar os dados dos experimentos realizados, mais uma vez observa-se que há um impacto significativo no desempenho do serviço quando se utiliza mecanismos de segurança. Um exemplo que pode ser verificado é o da comparação do Web service que não usa mecanismos de segurança, e cujo tempo médio é de 2,22 milissegundos, com o Web service que faz o uso do WS-Security, para assinatura e criptografia, cujo tempo médio é de 395,17 milissegundos. Ou seja, o que usa os mecanismos de segurança é aproximadamente 178 vezes mais lento.

%{\color{red}essa tabela est\'{a} muito ruim. sugiro remover}

%\begin{table}[h]
%\centering
%\begin{tabular}{|l|l|}
%\hline
%\textbf{Informação} & \textbf{Valor} \\ \hline
%Nº de Requisições   &  5000              \\ \hline
%Média               &  4677             \\ \hline
%Mínimo              &  118          \\ \hline
%Máximo              &  21665        \\ \hline
%Desvio Padrão       &  3170,49      \\ \hline
%\% Erro             &  0,0          \\ \hline
%Vazão/seg           &  8,62     \\ \hline
%Kbps                &  11,8         \\ \hline
%Média de Bytes      &  1401,05      \\ \hline
%\end{tabular}
%5\caption {Estatística básica com a utilização do protocolo de autenticação e autorização para 50 usuários.}\label{tb:estatistica_com_cripto_50}
%\end{table}

Ao analisar os resultados dos estudos relatados nesta seção e compará-los aos resultados obtidos com a análise de desempenho do Protocolo de Autenticação e Autorização proposto, percebe-se que os tempos médios de resposta do Protocolo estão dentro de um padrão aceitável, apesar do impacto da incorporação dos mecanismos de segurança.

\subsection{S\'{i}ntese da An\'{a}lise de Desempenho}

Neste cap\'{i}tulo foi realizada uma análise de desempenho, onde ficou evidente que o impacto no desempenho é significativo quando comparado com a não utilização da proposta. Porém os tempos médios obtidos estão dentro de um tempo médio esperado, que é abaixo de 5 segundos. Observou-se ainda, que em um cenário extremo de utilização, onde a PCDF tenha que atender a uma média de 2 duas requisições por segundo, e um período de 1 uma hora, ou seja 1200 requisições, a arquitetura proposta atende perfeitamente, pois consegue atender a 8,62 requisições por segundo em um cenário extremo onde foram realizados 5000 requisições.

  % inserir demais capítulos


  \postextual
  \bibliographystyle{plain}
  %\bibliography{bibliografia}
  %\bibliography{bibliografia2}
  \bibliography{reflatex}

%\appendix
%  \input{tex/Anexo1}

\end{document}
