\chapter{Conclusão}%

A Polícia Civil do Distrito Federal tem a necessidade de integrar e compartilhar informações sensíveis com órgãos conveniados. Devido à criticidade dessas informações, preocupações relacionadas à segurança foram levantadas e devem ser tratadas sob uma perspectiva arquitetural.

Nesse sentido, este trabalho teve como objetivo propor um Protocolo de Autenticação e Autorização seguro, baseado no modelo desafio-resposta, que possibilitasse a integração segura das informações disponibilizadas por meio de uma arquitetura orientada a serviços. O protocolo foi desenvolvido aderente à arquitetura REST e incorpora mecanismos de segurança tais como: criptografia, assinatura digital e o uso de certificados digitais. Dessa forma, ele atende aos requisitos básicos de segurança: autenticidade, integridade, confidencialidade e disponibilidade.

Um ponto importante no desenvolvimento do trabalho foi a utilização da lógica BAN, para analisar formalmente o protocolo proposto, haja vista que ela é uma lógica pioneira utilizada para analisar protocolos criptográficos. Com a realização deste trabalho foi possível ampliar os conhecimentos sobre SOA, mecanismos de segurança e integração utilizando serviços web. Além disso, também foi possível realizar análises de desempenho e de segurança do protocolo. Em ambos os casos o protocolo se mostrou confiável e viável.

A utilização do Protocolo de Autenticação e Autorização proposto, por parte da PCDF, possibilitará que as informações críticas que antes não eram compartilhadas por não haver formas seguras de compartilhamento das informações dentro da Instituição, possam agora ser compartilhadas e efetivamente integradas de forma eficiente e principalmente segura com os órgãos parceiros.

A solução proposta neste trabalho vai ao encontro das necessidades da PCDF, e pode ser utilizada por qualquer órgão de segurança, que da mesma forma, tenha a necessidade de compartilhar e integrar informações sensíveis de modo seguro.
\section{Contribuições}
A principal contribuição deste trabalho é a definição de um protocolo de autenticação e autorização seguro, desenvolvido para permitir a integração e o compartilhamento de informações sigilosas. O protocolo proposto é aderente à arquitetura REST, incorpora boas prática de segurança, possibilitando a autenticação e autorização de usuários de forma segura. Dessa forma, no contexto da sua utilização no âmbito da PCDF, ele permitirá que os serviços disponibilizados pela instituição possam ser ofertados e efetivamente integrados aos sistemas de órgãos parceiros. Além disso, foi possível avaliar a importância da utilização de métodos formais no planejamento e verificação de protocolos criptográficos. %Uma vez que com a formalização do protocolo, uma avaliação mais detalhada possibilita identificar se os objetivos de segurança propostos são obtidos.

Outra contribuição foi a implementação de um protótipo desenvolvido utilizando a linguagem de programação funcional Haskell. O protocolo está disponível para qualquer órgão que deseje utilizá-lo no processo de integração. O desenvolvimento do protótipo contou com a contribuição direta de um trabalho de conclusão de curso, do aluno Alexandre Lucchesi, o que serviu para estabelecer uma colaboração entre o Mestrado Profissional em Computação Aplicada com alunos da Engenharia da Computação.

A avaliação de desempenho, que foi realizada a partir do protótipo implementado, também constituiu em uma importante contribuição, haja vista que com sua realização podemos verificar qual impacto que a utilização de mecanismos de segurança tem sobre arquiteturas orientadas a serviço.

Outras contribuições que devem ser ressaltadas são o mapeamento sistemático e a  revisão da literatura, apresentados nos Capítulos 2 e 3 respectivamente. O mapeamento sistemático possibilitou aprofundamento dos conhecimentos referentes à segurança em SOA.  A revisão da literatura sintetizou os principais conceitos relacionados a SOA, sendo possível verificar entre outros pontos, quais são os padrões para construção de software seguro em SOA.



\section{Trabalhos Futuros}

Quanto a trabalhos futuros, vislumbramos a possibilidade de implementar otimizações ao Protocolo de Autenticação e Autorização proposto, de forma que ele possa ser melhorado sob o aspecto de desempenho e segurança.

Além disso, também seria importante a realização de estudos mais aprofundados sobre possíveis vulnerabilidades tanto do lado do cliente, consumidor do serviço, quanto do lado da PCDF, provedora dos serviços.

Outro ponto que também pode ser proposto como trabalhos futuros dizem respeito à realização de mais experimentos que visem comparar o desempenho do protocolo com a utilização de diferentes algoritmos de criptografia.





